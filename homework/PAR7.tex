\documentclass[letter]{article}
\usepackage{amsmath}
\usepackage{amsfonts}
\usepackage{amssymb}
\usepackage{ifthen}
\usepackage{fancyhdr}
\usepackage{enumitem}
\usepackage[usenames,dvipsnames,svgnames,table]{xcolor}

%%%
% Set up the margins to use a fairly large area of the page
%%%
\oddsidemargin=.2in
\evensidemargin=.2in
\textwidth=6in
\topmargin=0in
\textheight=9.0in
\parskip=.07in
\parindent=0in
\pagestyle{fancy}

\expandafter\def\expandafter\quote\expandafter{\quote\sf\color{DarkGreen}}

%%%
% Set up the header
%%%
\newcommand{\setheader}[6]{
	\lhead{{\sc #1}\\{\sc #2} ({\small \it \today})}
	\rhead{
		{\bf #3} 
		\ifthenelse{\equal{#4}{}}{}{(#4)}\\
		{\bf #5} 
		\ifthenelse{\equal{#6}{}}{}{(#6)}%
	}
}

%%%
% Set up some shortcut commands
%%%
\newcommand{\R}{\mathbb{R}}
\newcommand{\N}{\mathbb{N}}
\newcommand{\Z}{\mathbb{Z}}
\newcommand{\Proj}{\mathrm{proj}}
\newcommand{\Perp}{\mathrm{perp}}
\newcommand{\proj}{\mathrm{proj}}
\newcommand{\Span}{\mathrm{span}}
\newcommand{\Null}{\mathrm{null}}
\newcommand{\Rank}{\mathrm{rank}}
\newcommand{\Det}{\mathrm{det}}
\newcommand{\mat}[1]{\begin{bmatrix}#1\end{bmatrix}}
\renewcommand{\d}{\mathrm{d}}

%%%
% This is where the body of the document goes
%%%
\begin{document}
	\setheader{Math 240}{PAR 7}{Due: Wednesday, November 23}{}{}{}
	Though this is called a ``PAR,'' we will not be doing the peer-assisted reflection
	portion of this assignment.

	We've come a long way in this course---we've gone from row-reduction to matrix equations, from
	spans and linear independence to eigenvectors and diagonalization.  It's now time to reflect
	on the term and see how we've changed.

	This homework involves writing two short essays.  Essay 1 should be at most one page and 
	essay 2 should be 1--2 pages.  Please label your essays appropriately.
	
	\begin{enumerate}
		\item Before the course started, you wrote a homework 0.  Please re-read what
			you wrote for homework 0 and explain whether or not your thoughts on
			any of the questions have been reinfoced or changed.  Please provide at least
			one example of why your opinions now are the same or different.  The example
			can come from this class, but it doesn't have to.
		
		
		\item 
			Imagine you are writing a letter to a parent/aunt/uncle who has 
			taken technical math/science courses in
			the past, but has forgotten most of the terminology.  Write to this audience explaining 
			what Math 240 is about, and highlight the main ideas in the class (when writing to 
			non-technical audience, examples are key).  In your essay, please do the following:
			\begin{enumerate}
				\item[(i)] address at least one
					of the three over-arching {\sc Learning Outcomes} listed on the course syllabus and 
					whether you feel you've achieved that learning outcome;
				\item[(ii)] include one technical math definition written in both math language and
					explained with plain English
				\item[(iii)] explain, for at least one linear algebra concept, how it can be viewed
					geometrically \emph{and} algebraically.
			\end{enumerate}

			Your essay should start on a new page (i.e.,
			essay 1 gets its own page(s) and this essay gets its own pages).  To help you set the mood, 
			start your letter with,
			\begin{quote}
				Dear Mom, 

				This term at Northwestern I've been taking Math 240, which is a course about$\ldots$
			\end{quote}

			Of course, if you're inspired to start your essay a different way, feel free to!
			You should have fun with this essay.  After all, how often do you get a chance to explain 
			to others and yourself what a university course really taught you?

			Some guidelines on how your essay will be evaluated:
			\begin{enumerate}
				\item[Acceptable:]
					Correctly identified a learning outcome from the syllabus 
					and conveyed to the reader whether or not you have achieved it; gave a definition in mathematical
					and plain language; gave an example of a concept that we view geometrically and algebraically.
				\item[Good:]
					Correctly identified a learning outcome from the syllabus
					and explained clearly \emph{why} you have or have not achieved it; gave a definition in mathematical
					and plain language and explained \emph{how}
					this definition relates to solving linear algebra problems; 
					gave an example of a concept that we view geometrically and algebraically.
				\item[Excellent:]
					Correctly identified a learning outcome from the syllabus, 
					discussed \emph{how} this learning outcome connects to specific material or themes from the course, 
					and explained clearly \emph{why} you have or have not achieved it;  gave a definition in mathematical
					and plain language and explained \emph{how}
					this definition relates to solving linear algebra problems and \emph{how} definitions play an
					important role in linear algebra; gave an example of a concept that we view geometrically and algebraically
					and explained \emph{how} having a geometric and algebraic viewpoint of the same concept
					is useful.
			\end{enumerate}

		
	\end{enumerate}

\end{document}
