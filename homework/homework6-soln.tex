\documentclass[letter]{article}
\usepackage{amsmath}
\usepackage{amsfonts}
\usepackage{amssymb}
\usepackage{amsthm}
\usepackage{mdframed}
\usepackage{ifthen}
\usepackage{fancyhdr}
\usepackage[usenames,dvipsnames,svgnames,table]{xcolor}
\usepackage{tikz}
\usepackage{changepage}
\graphicspath{{images/}}
\usepackage[hidelinks]{hyperref}
\usepackage{mathrsfs}

%%%
% Set up the margins to use a fairly large area of the page
%%%
\oddsidemargin=.2in
\evensidemargin=.2in
\textwidth=6in
\topmargin=0in
\parskip=.07in
\parindent=0in
\pagestyle{fancy}

\expandafter\def\expandafter\quote\expandafter{\quote\sf\color{DarkGreen}}

%%%
% Set up the header
%%%
\newcommand{\setheader}[7]{
	\lhead{{\sc #1}\\{\sc #2} %({\small \it \today})
	}
	\rhead{
		{\bf #3} 
		\ifthenelse{\equal{#4}{}}{}{(#4)}\\
		{\bf #5} 
		\ifthenelse{\equal{#6}{}}{#7}{(#6)}%
	}
}

%%%
% Set up closure command
%%%
\newcommand{\overbar}[1]{\mkern 1.5mu\overline{\mkern-1.5mu#1\mkern-1.5mu}\mkern 1.5mu}
%%%
% Set up some shortcut commands
%%%
\newcommand{\R}{\mathbb{R}}
\newcommand{\N}{\mathbb{N}}
\newcommand{\Z}{\mathbb{Z}}
\newcommand{\Q}{\mathbb{Q}}
\newcommand{\Proj}{\mathrm{proj}}
\newcommand{\Perp}{\mathrm{perp}}
\newcommand{\Span}{\mathrm{span}}
\newcommand{\Null}{\mathrm{null}}
\newcommand{\Rank}{\mathrm{rank}}
\newcommand{\Range}{\mathrm{range}}
\newcommand{\Col}{\mathrm{col}}
\newcommand{\Row}{\mathrm{row}}
\newcommand{\Det}{\mathrm{det}}
\newcommand{\mat}[1]{\begin{bmatrix}#1\end{bmatrix}}
\newcommand{\Rref}{\mathrm{rref}}
\renewcommand{\d}{\mathrm{d}}
\newcommand{\abs}[1]{\left\vert #1 \right\vert}
\newcommand{\norm}[1]{\left\| #1 \right\|}
\newcommand{\inner}[2]{\langle #1, #2 \rangle}
\newcommand{\RI}{\mathscr{R}}
\newcommand{\RSI}[1]{\mathscr{R}(#1)}

\DeclareMathOperator{\arccot}{arccot}
\DeclareMathOperator{\diam}{diam}

\def\upint{\mathchoice%
    {\mkern13mu\overline{\vphantom{\intop}\mkern7mu}\mkern-20mu}%
    {\mkern7mu\overline{\vphantom{\intop}\mkern7mu}\mkern-14mu}%
    {\mkern7mu\overline{\vphantom{\intop}\mkern7mu}\mkern-14mu}%
    {\mkern7mu\overline{\vphantom{\intop}\mkern7mu}\mkern-14mu}%
  \int}
\def\lowint{\mkern3mu\underline{\vphantom{\intop}\mkern7mu}\mkern-10mu\int}

\newenvironment{answer}{
	\begin{adjustwidth}{8mm}{} \vspace{2mm}}{\end{adjustwidth} \vspace{2mm}
}

\theoremstyle{plain}
\newtheorem*{theorem}{Theorem}
\newtheorem{lemma}{Lemma}
\newtheorem*{prop}{Proposition}

\theoremstyle{definition}
\newtheorem*{definition}{Definition}

\theoremstyle{remark}
\newtheorem*{claim}{Claim}

%%%
% This is where the body of the document goes
%%%
\begin{document}
	\setheader{Math 240}{Homework 6 Solutions}{Due: Thursday, November 17}{}{}{}{}
	\emph{We are nearly experts at typing by now.  Though you are not required
	to type your homework, it's strongly encouraged.  You can even download
	the \texttt{tex} file for this homework and type your answers below
	each problem.  Using the} \verb!\begin{quote}! \emph{and} \verb!\end{quote}! \emph{environment
		will indent anything you type inbetween.  Perfect for typing answers!}
	
	\begin{enumerate}
		
		%\item Let $\vec{u} = \mat{1\\1\\1}$, $\vec{v} = \mat{1\\-1\\-1}$, $\vec{w} = \mat{-1\\1\\-1}$, and $\vec{x} = \mat{2\\0\\0}$.
		%\begin{enumerate}
		%	\item Find a linear transformation $T_1$ such that $T_1(\vec{e_1}) = \mat{1\\2\\3}$, $T_1(\vec{e_2}) = \mat{4\\5\\6}$, and $T_1(\vec{e_3}) = \mat{7\\8\\9}$.
		%	\item Find a linear transformation $T_2$ such that $T_2(\vec{u}) = \mat{1\\2\\3}$, $T_2(\vec{v}) = \mat{4\\5\\6}$, and $T_2(\vec{w}) = \mat{7\\8\\9}$.
		%	\item Is there a linear transformation $T_3$ such that $T_3(\vec{u}) = \mat{1\\2\\3}$, $T_3(\vec{v}) = \mat{4\\5\\6}$, and $T_3(\vec{x}) = \mat{7\\8\\9}$?  Explain why or why not.
		%\end{enumerate}
		%You may give your answers in the form of a matrix.
		
		\item Let $A = \mat{1 & 0 & 1 & 1 & 3 \\ 1 & 2 & 3 & -1 & -1 \\ 3 & 1 & 4 & 2 & 7}$.  Let $\mathcal{C}$, $\mathcal{R}$, and $\mathcal{N}$ be the column, row, and null spaces of $A$, respectively.
		\begin{enumerate}
			\item Find a basis for $\mathcal{C}$
			\begin{answer}
				We begin with row reduction. 
				\[
				\left[\begin{array}{ccccc}
					1 & 0 & 1 & 1 & 3 \\
					1 & 2 & 3 & -1 & -1 \\
					3 & 1 & 4 & 2  & 7
				\end{array}\right] \sim 
				\left[\begin{array}{ccccc}
					1 & 0 & 1 & 1 & 3 \\
					0 & 2 & 2 & -2 & -4 \\
					0 & 1 & 1 & -1 & -2
				\end{array}\right] \sim 
				\left[\begin{array}{ccccc}
					1 & 0 & 1 & 1 & 3 \\
					0 & 1 & 1 & -1 & -2 \\
					0 & 0 & 0 & 0 & 0
				\end{array}\right].
				\]
				Since $\Rref(A)$ has two pivot columns, columns $1$ and $2$, a basis for the column space of $A$ is
				\[
				\left\{\mat{1\\1\\3}, \mat{0\\2\\1}\right\}.
				\]
			\end{answer}
			\item Find a basis for $\mathcal{R}$.
			\begin{answer}
				Since $\Rref(A)$ has two non-zero rows, rows $1$ and $2$, a basis for the row space of $A$ is either
				\[
				\left\{\mat{1\\0\\1\\1\\3}, \mat{0\\1\\1\\-1\\-2}\right\} \text{ or } \left\{\mat{1\\0\\1\\1\\3}, \mat{1\\2\\3\\-1\\-1}\right\}.
				\]
			\end{answer}
			\item Find a basis for $\mathcal{N}$.
			\begin{answer}
				In order to find a basis for $\mathcal{N}$, we begin by solving the matrix equation
				\[
				\left[\begin{array}{ccccc}
					1 & 0 & 1 & 1 & 3 \\
					0 & 1 & 1 & -1 & -2 \\
					0 & 0 & 0 & 0 & 0
				\end{array}\right] \mat{x_1\\x_2\\x_3\\x_4\\x_5} = 
				\mat{0\\0\\0\\0\\0}.
				\]
				Therefore, we have that
				\[
				\begin{cases}
					x_1 + x_3 + x_4 + 3x_5 &= 0\\
					x_2 + x_3 - x_4 - 2x_5 &= 0
				\end{cases} \iff
				\begin{cases}
					x_1 &= -x_3 - x_4 - 3x_5 \\
					x_2 &= -x_3 + x_4 + 2x_5
				\end{cases}.
				\]
				Therefore, the null space given in vector form is
				\[
				\mat{x_1\\x_2\\x_3\\x_4\\x_5} = x_3\mat{-1\\-1\\1\\0\\0} + x_4\mat{-1\\1\\0\\1\\0} + x_5\mat{-3\\2\\0\\0\\1}.
				\]
				Hence, a basis for $\Null(A)$ is 
				\[
				\left\{\mat{-1\\-1\\1\\0\\0}, \mat{-1\\1\\0\\1\\0}, \mat{-3\\2\\0\\0\\1}\right\}.
				\]
			\end{answer}
			\item Find the dot product of each of your basis vectors for $\mathcal{R}$ with each of your basis vectors for $\mathcal{N}$.
			What can you say geometrically about $\mathcal{R}$ and $\mathcal{N}$?
			\begin{answer}
				Observe that $[1,0,1,1,3] \cdot [-1,-1,1,0,0] = -1 + 0 + 1+ 0 + 0 = 0$. Similarly, the dot products of each other basis vector of $\mathcal{R}$ dotted with each other basis vector for $\mathcal{N}$ are all $0$. Hence, $\mathcal{R}$ and $\mathcal{N}$ are perpendicular. 
			\end{answer}
		\end{enumerate}
		
		\item Let $L$ be the line $x=y=z$ in $\R^3$.
		\begin{enumerate}
			\item Find a $3\times 3$ matrix $B_1$ whose column space is the $xy$-plane and whose null space is $L$.
			\begin{answer}
				Let
				\[
				B_1 = \left[\begin{array}{ccc}
							a & b & c \\
							d & e & f \\
							g & h & l
						\end{array}\right].
				\]
				Since the column space is the $xy$-plane, we have that $g=h=l=0$. Next, we seek to find $\Null(B_1)$. To do so, we solve the matrix equation
				\[
				\left[\begin{array}{ccc}
					a & b & c \\
					d & e & f \\
					0 & 0 & 0
				\end{array}\right] \mat{x\\y\\z} = \mat{0\\0\\0} \iff
				\begin{cases}
					ax+by+cz &= 0 \\
					dx+ey+fz &= 0 
				\end{cases}. 
				\]
				Further note that $\Null(B_1)$ is a line, and thus has dimension $1$. Therefore, $\Row(B_1)$ must have dimension $2$, and thus is a plane. We also know that $\Row(B_1)$ and $\Null(B_1)$ are perpendicular, and that $\Row(B_1)$ is 
				\[
				\Span \left\{\mat{a\\b\\c}, \mat{d\\e\\f}\right\}.
				\]
				Thus, we need to find $2$ linearly independent vectors that are perpendicular to the direction vector of the line $x=y=z$. This direction vector is $[1,1,1]$. Thus, we pick 
				\[
				\mat{a\\b\\c} = \mat{1\\-1\\0} \text{ and } \mat{d\\e\\f} = \mat{1\\0\\-1} \implies B_1 = \mat{1 & -1 & 0 \\ 1 & 0 & -1 \\ 0 & 0 & 0}.
				\]
			\end{answer}
			\item Find a $3\times 3$ matrix $B_2$ whose column space is the $xz$-plane and whose null space is $L$.
			\begin{answer}
				By a similar idea to part (a), we find that a possible $B_2$ is
				\[
				\mat{1 & -1 & 0 \\ 0 & 0 & 0 \\ 1 & 0 & -1}.
				\]
			\end{answer}
			\item Find the row space of $B_1$ and the row space of $B_2$.  How do they compare?  Explain.
			\begin{answer}
				Both $\Row(B_1)$ and $\Row(B_2)$ are equal to
				\[
				\Span\left\{\mat{1\\-1\\0}, \mat{1\\0\\-1}\right\}.
				\]
				Thus, they are identical. This is further evidenced by the fact that since $\Null(B_1)$ and $\Null(B_2)$ have dimension $1$, and thus $\Row(B_1)$ and $\Row(B_2)$ both have dimension $2$. Further, both row spaces are perpendicular to the line $x=y=z$. Since the plane that goes through the origin and is perpendicular to the line $x=y=z$ is unique, we have that $\Row(B_1) = \Row(B_2)$. 
			\end{answer}
		\end{enumerate}
		
		\item A linear transformation $T$ has the following effects:
		\begin{itemize}
			\item Along the line $y = x$, it ``shrinks'' everything by a factor of 2---points along this line move halfway to the origin.
			\item Along the line $y=-3x$, it reflects everything over the origin---if $\vec{v}$ is the position vector of a point on this line, then it moves to $-\vec{v}$.
		\end{itemize}
		\begin{enumerate}
			\item Find a matrix $A$ such that $T(\vec{v}) = A\vec{v}$ for all $\vec{v} \in \R^2$.
			\begin{answer}
				First, we pick a vector on the line $y=x$. Such a vector is $[1,1]$. Therefore, we know that 
				\[
				A \mat{1\\1} = \mat{1/2 \\ 1/2}. 
				\]
				Further, we pick a vector on the line $y = -3x$. Such a vector is $[1,-3]$. Then, we know that
				\[
				A \mat{1\\-3} = \mat{-1\\3}.
				\]
				Let
				\[
				A = \mat{a & b \\ c & d}.
				\]
				We set up a system of equations using the above vectors
				\[
				\begin{cases}
					a + b &= 1/2 \\
					c + d &= 1/2 \\
					a - 3b &= -1 \\
					c - 3d &= 3
				\end{cases} \iff
				\begin{cases}
					a &= 1/8 \\
					b &= 3/8 \\
					c &= 9/8 \\
					d &= 5/8
				\end{cases} \implies
				A = \mat{1/8 & 3/8 \\ 9/8 & -5/8}.
				\]
			\end{answer}
			\item Find $A^{-1}$.
			\begin{answer}
			We row reduce an augmented matrix.
			\[
			[A \mid I] = \left[\begin{array}{cc|cc}
				1/8 & 3/8 & 1 & 0 \\
				9/8 & -5/8 & 0 & 1
			\end{array}\right] \sim
			\left[\begin{array}{cc|cc}
				1 & 3 & 8 & 0 \\
				0 & -32 & -72 & 8
			\end{array}\right] \sim 
			\left[\begin{array}{cc|cc}
				1 & 3 & 8 & 0 \\
				9 & -5 & 0 & 8
			\end{array}\right]
			\]
			Further reduction gives
			\[
			\sim \left[\begin{array}{cc|cc}
				1 & 3 & 8 & 0 \\
				9 & -5 & 0 & 8
			\end{array}\right] \sim 
			\left[\begin{array}{cc|cc}
				1 & 0 & 5/4 & 3/4 \\
				0 & 1 & 9/4 & -1/4
			\end{array}\right] \implies 
			A^{-1} = \mat{5/4 & 3/4 \\ 9/4 & -1/4}.
			\]
			\end{answer}
			\item What are the ``stretch factors'' and ``stretch directions'' of the transformation given by $A^{-1}$?
			\begin{answer}
				If $\lambda \neq 0$ is an eigenvalue for $A$ with corresponding vector $\vec{x}$, then $A\vec{x} = \lambda \vec{x} \implies A^{-1} A \vec{x} = A^{-1} \lambda \vec{x}$. Therefore, $\vec{x} = \lambda A^{-1} \vec{x}$. Thus, 
				\[
				\frac{1}{\lambda} \vec{x} = A^{-1} \vec{x}.
				\]
				Thus, $\vec{x}$ is also an eigenvector for $A^{-1}$, with corresponding eigenvalue $1/\lambda$. In this problem $A^{-1}$ has stretch factor $1/2$ along the direction $[1,1]$ and a stretch factor of $-1$ along the direction $[1,-3]$. 
			\end{answer}
		\end{enumerate}

		%\item Let $T:\R^n\to\R^n$. A subspace $X\subseteq \R^n$ is called \emph{invariant with respect to $T$} if $TX=X$.  That is,
			%$\{\vec v:\vec v=T\vec x\text{ for some }\vec x\in X\}=X$.  Note, this is \emph{different} than saying $T\vec x=\vec x$ for all $\vec x\in X$.
		%\begin{enumerate}
			%\item Describe all invariant subspaces of the linear transformation given by the matrix $\mat{1&0\\0&2}$.
			%\item Describe all invariant subspaces of the linear transformation given by the matrix $\mat{2&0\\0&2}$.
			%\item Describe all invariant subspaces of the sheer given by the matrix $\mat{1&a\\0&1}$ where $a\neq 0$.
			%\item Explain how invariant subspaces relate to eigenvectors.
		%\end{enumerate}
		
		\item Recall the ``italicizing $N$'' matrix that you found in class: $A = \mat{1 & \frac{1}{3} \\ 0 & \frac{4}{3}}$.
		Find all eigenvalues (``stretch factors'') and eigenvectors (``stretch directions'') of $A$.
		\begin{answer}
			Observe that $A$ is a triangular matrix, and therefore its eigenvalues are the entries on its main diagonal: $\lambda = 1$ and $\lambda = 4/3$. 
			\begin{itemize}
				\item $\lambda = 1$. 
				\[
				(A - \lambda I) \mat{x\\y} = \mat{0 & 1/3 \\ 0 & 1/3}\mat{x\\y} = \mat{0\\0} \iff \frac{1}{3}y = 0 \iff y=0.
				\]
				Thus, the eigenvectors are vectors of the form 
				\[
				\mat{x\\0} = x\mat{1\\0}
				\]
				with $x \neq 0$. 
				\item $\lambda = 4/3$
				\[
				(A - \lambda I) \mat{x\\y} = \mat{-1/3 & 1/3 \\ 0 & 0}\mat{x\\y} = \mat{0\\0} \iff -\frac{1}{3} x + \frac{1}{3} y = 0 \iff y = x.
				\]
				Thus, the eigenvectors are vectors of the form
				\[
				\mat{x\\x} = x\mat{1\\1}
				\]
				with $x \neq 0$. 
			\end{itemize}
		\end{answer}
		
		%\item Give an example of a 2-by-2 matrix with fewer than two ``stretch directions''.  (That is, find a matrix with fewer than two linearly indepedent eigenvectors.)
			
		
		
			
		
	\end{enumerate}

\end{document}
