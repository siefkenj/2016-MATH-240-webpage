\documentclass[letter]{article}
\usepackage{amsmath}
\usepackage{amsfonts}
\usepackage{amssymb}
\usepackage{ifthen}
\usepackage{fancyhdr}

\usepackage{enumitem}

%%%
% Set up the margins to use a fairly large area of the page
%%%
\oddsidemargin=.2in
\evensidemargin=.2in
\textwidth=6in
\topmargin=0in
\textheight=9.0in
\parskip=.07in
\parindent=0in
\pagestyle{fancy}

%%%
% Set up the header
%%%
\newcommand{\setheader}[6]{
	\lhead{{\sc #1}\\{\sc #2} ({\small \it \today})}
	\rhead{
		{\bf #3} 
		\ifthenelse{\equal{#4}{}}{}{(#4)}\\
		{\bf #5} 
		\ifthenelse{\equal{#6}{}}{}{(#6)}%
	}
}

%%%
% Set up some shortcut commands
%%%
\newcommand{\R}{\mathbb{R}}
\newcommand{\N}{\mathbb{N}}
\newcommand{\Z}{\mathbb{Z}}
\newcommand{\Proj}{\mathrm{proj}}
\newcommand{\Perp}{\mathrm{perp}}
\newcommand{\proj}{\mathrm{proj}}
\newcommand{\Span}{\mathrm{span}}
\newcommand{\Null}{\mathrm{null}}
\newcommand{\Rank}{\mathrm{rank}}
\newcommand{\mat}[1]{\begin{bmatrix}#1\end{bmatrix}}
\newcommand{\sbv}[1]{\ensuremath{\mathbf{e}_{#1}}}

%%%
% This is where the body of the document goes
%%%
\begin{document}
	\setheader{Math 240}{Homework 1}{Due: Thursday, September 29th}{}{}{}
	Problems 1--3 should be review. \emph{Review} is not
	the same as \emph{easy}; it just means that the problems don't require knowledge beyond
	Math 230.  If you feel rusty, please refresh yourself on vectors, dot products, and
	planes before starting.  Note that $\sbv{1}$ and $\sbv{2}$ are unit vectors
	pointing along the $x$ and $y$ axes respectively.  They might have been called $\mathbf{i}$
	and $\mathbf{j}$ in your Math 230 class.
	\begin{enumerate}
		\item Find unit vectors in the directions $\vec u=\mat{2\\-1}$,
		$\vec v=\mathbf{e}_1 + 3 \mathbf{e}_2$, $\vec w=-\sbv{1} - \sbv{2} - \sbv{3}$, and
		$\vec r=\frac{\vec v}{\|\vec v\|}+\frac{\vec w}{\|\vec w\|}$.

		\item The dot product of vectors (in $\R^n$) can be defined algebraically by:
		\[
			\mat{a_1\\a_2\\\vdots\\a_n} \cdot \mat{b_1\\b_2\\\vdots\\b_n} = a_1 b_1 + a_2 b_2 + \cdots + a_n b_n
		\]
		
		\begin{enumerate}
			\item Rewrite this definition in summation notation.
		\end{enumerate}
		Use the algebraic definition of the dot product to show that for any 
		$\vec a,\vec b,\vec c\in\R^n$ and $\alpha\in \R$:
		\begin{enumerate}[resume]
			\item $\vec a\cdot (\alpha\vec b) = \alpha(\vec a\cdot \vec b)$
			\item $\vec a\cdot(\vec b+\vec c) = \vec a\cdot \vec b+\vec a\cdot \vec c$
		\end{enumerate}


		\item Consider the plane $\mathcal P$ defined as the set of solutions to the
		equation 
		\[
			3x-2y+z=4.
		\]
		Let $\vec p\in \mathcal P$ be a point in $\mathcal P$ interpreted as a vector.
		\begin{enumerate}
			\item What is $\vec p\cdot \mat{3\\-2\\1}$?  Explain.
			\item The plane $\mathcal Q$ is the plane $\mathcal P$ translated 
			in the $y$-direction (or the $\sbv{2}$-direction) by one unit.  Let $\vec q\in\mathcal Q$.
			What is $\vec q\cdot\mat{3\\-2\\1}$?
		\end{enumerate}

		\item Sketch each of the following sets on a number line or in the coordinate plane:
		\begin{enumerate}
			\item $\{x \in \R \,:\, 2x \in \Z\}$
			\item $\{(x,y) \in \R^2 \,:\, x \le 2y \}$
		\end{enumerate}
		Write each of the following sets in set-builder notation:
		\begin{enumerate}[resume]
			\item The graph of $y=x^2$
			\item The filled-in triangle with vertices $(0,0)$, $(1,0)$, and $(0,1)$
		\end{enumerate}
			
		\item A \textbf{linear combination} of two vectors $\vec{u}$ and $\vec{v}$ is any vector of the form 
		\[
			a\vec{u} + b\vec{v}
		\]
		where $a$ and $b$ are scalars.
		\begin{enumerate}
			\item Write $\mat{107\\64}$ as a linear combination of $\mat{3\\1}$ and $\mat{1\\2}$.
			\item Write $\mat{-99\\-100}$ as a linear combination of $\mat{3\\1}$ and $\mat{1\\2}$.
			\item Linear combinations of three vectors are defined similarly.  Which vectors in $\R^3$ can be written as a linear combination of $\mat{1\\0\\1}$, $\mat{2\\0\\0}$, and $\mat{0\\0\\-1}$?  Explain how you know.
		\end{enumerate}

		
	\end{enumerate}
	
	\textbf{Don't forget to also complete PAR1, this week's Peer-Assisted Reflection problems.}
	




\end{document}
