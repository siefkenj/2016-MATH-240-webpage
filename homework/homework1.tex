\documentclass[letter]{article}
\usepackage{amsmath}
\usepackage{amsfonts}
\usepackage{amssymb}
\usepackage{ifthen}
\usepackage{fancyhdr}

%%%
% Set up the margins to use a fairly large area of the page
%%%
\oddsidemargin=.2in
\evensidemargin=.2in
\textwidth=6in
\topmargin=0in
\textheight=9.0in
\parskip=.07in
\parindent=0in
\pagestyle{fancy}

%%%
% Set up the header
%%%
\newcommand{\setheader}[6]{
	\lhead{{\sc #1}\\{\sc #2} ({\small \it \today})}
	\rhead{
		{\bf #3} 
		\ifthenelse{\equal{#4}{}}{}{(#4)}\\
		{\bf #5} 
		\ifthenelse{\equal{#6}{}}{}{(#6)}%
	}
}

%%%
% Set up some shortcut commands
%%%
\newcommand{\R}{\mathbb{R}}
\newcommand{\N}{\mathbb{N}}
\newcommand{\Z}{\mathbb{Z}}
\newcommand{\Proj}{\mathrm{proj}}
\newcommand{\Perp}{\mathrm{perp}}
\newcommand{\proj}{\mathrm{proj}}
\newcommand{\Span}{\mathrm{span}}
\newcommand{\Null}{\mathrm{null}}
\newcommand{\Rank}{\mathrm{rank}}
\newcommand{\mat}[1]{\begin{bmatrix}#1\end{bmatrix}}

%%%
% This is where the body of the document goes
%%%
\begin{document}
	\setheader{Math 281-1}{Homework 1}{Due: Thursday, October 1}{}{}{}
	\begin{enumerate}
		\item Find unit vectors in the directions $\vec u=\mat{2\\-1}$,
		$\vec v=\hat x+3\hat y$, $\vec w=-\hat x-\hat y-\hat z$, and
		$\vec r=\frac{\vec v}{\|\vec v\|}+\frac{\vec w}{\|\vec w\|}$.

		\item Use the algebraic definition of the dot product to show for any 
		$\vec a,\vec b,\vec c\in\R^n$ and $\alpha\in \R$
		\begin{enumerate}
			\item[(i)] $\vec a\cdot (\alpha\vec b) = \alpha(\vec a\cdot \vec b)$
			\item[(ii)] $\vec a\cdot(\vec b+\vec c) = \vec a\cdot \vec b+\vec a\cdot \vec c$
		\end{enumerate}
		(See page 7 of the Evans text for the definition of $\alpha\vec v$ and $\vec u+\vec v$.)

		\item For vectors $\vec a,\vec b,\vec c\in\R^n$, is $\vec a\cdot\vec b\cdot \vec c$ defined?
		Explain your answer in terms of the definition of the dot product.

		\item Consider the plane $\mathcal P$ defined as the set of solutions to the
		equation 
		\[
			3x-2y+z=4.
		\]
		Let $\vec p\in \mathcal P$ be a point in $\mathcal P$ interpreted as a vector.
		\begin{enumerate}
			\item What is $\vec p\cdot \mat{3\\-2\\1}$.  Explain.
			\item The plane $\mathcal Q$ is the plane $\mathcal P$ translated 
			in the $\hat y$ direction by one unit.  Let $\vec q\in\mathcal Q$.
			What is $\vec q\cdot\mat{3\\-2\\1}$?
		\end{enumerate}

		\item Let $\vec r=\mat{1.5\\1}$ and consider the sets
		\[
			A=\{\vec x\in\R^2:\vec x=2t\vec r\text{ for some }t\in\Z\}
			\qquad
			B=\{\vec x\in\R^2:\vec x=3t\vec r\text{ for some }t\in\Z\}
		\]
		\[
			C=\{\vec x\in\R^2:\vec x=\vec a+\vec b\text{ for some }\vec a\in A\text{ and }\vec b\in B\}.
		\]
		\begin{enumerate}
			\item On separate axes, draw $A$, $B$, and $C$.
			\item Prove or disprove the following statements:
			\begin{enumerate}
				\item[(i)] $A=\{\vec x\in\R^2:\vec x=t\vec r\text{ for some }t\in\Z\}$
				\item[(ii)] $C=\{\vec x\in\R^2:\vec x=t\vec r\text{ for some }t\in\Z\}$
				\item[(iii)] $C=\{\vec x\in\R^2:\vec x=t\vec r\text{ for some }t\in\R\}$
			\end{enumerate}
			Make sure to include any relevant definitions. (\emph{Hint:} you may take
			it as a fact that if $a,b\in \Z$ then $a\pm b\in\Z$ and $ab\in \Z$.)
		\end{enumerate}
	\end{enumerate}
\end{document}
