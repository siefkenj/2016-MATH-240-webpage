\documentclass[letter]{article}
\usepackage{amsmath}
\usepackage{amsfonts}
\usepackage{amssymb}
\usepackage{ifthen}
\usepackage{fancyhdr}
\usepackage{enumitem}

%%%
% Set up the margins to use a fairly large area of the page
%%%
\oddsidemargin=.2in
\evensidemargin=.2in
\textwidth=6in
\topmargin=0in
\textheight=9.0in
\parskip=.07in
\parindent=0in
\pagestyle{fancy}

%%%
% Set up the header
%%%
\newcommand{\setheader}[6]{
	\lhead{{\sc #1}\\{\sc #2} ({\small \it \today})}
	\rhead{
		{\bf #3} 
		\ifthenelse{\equal{#4}{}}{}{(#4)}\\
		{\bf #5} 
		\ifthenelse{\equal{#6}{}}{}{(#6)}%
	}
}

%%%
% Set up some shortcut commands
%%%
\newcommand{\R}{\mathbb{R}}
\newcommand{\N}{\mathbb{N}}
\newcommand{\Z}{\mathbb{Z}}
\newcommand{\Proj}{\mathrm{proj}}
\newcommand{\Perp}{\mathrm{perp}}
\newcommand{\proj}{\mathrm{proj}}
\newcommand{\Span}{\mathrm{span}}
\newcommand{\Null}{\mathrm{null}}
\newcommand{\Rank}{\mathrm{rank}}
\newcommand{\Det}{\mathrm{det}}
\newcommand{\mat}[1]{\begin{bmatrix}#1\end{bmatrix}}
\renewcommand{\d}{\mathrm{d}}

%%%
% This is where the body of the document goes
%%%
\begin{document}
	\setheader{Math 240}{Homework 7}{Not for turning in}{}{}{}

	These problems will not be turned in nor graded.  However, a non-empty subset of these
	problems will appear on the final skills check/takehome.  The numbers may change when
	they appear on the exam, but it behooves you to have a thorough understanding of every problem on this homework.
	
	\begin{enumerate}
		
		\item Suppose the matrix equation $A\vec x=\mat{3\\2\\7}$ has the general solution
			\[
				\vec x=\mat{1\\0\\0}+s\mat{1\\1\\0}+t\mat{-1\\0\\1}.
			\]
			\begin{enumerate}
				\item How many rows and how many columns does $A$ have?
				\begin{quote}
					Based on the size of the input and output vectors, $A$ must be a $3\times 3$ matrix
				\end{quote}

				\item Find $\Null(A)$.
				\begin{quote}
					The complete solution takes the form of $\vec a+\Null(A)$ where $\vec a$ is a particular
					solution.  Therefore $\Null(A)=\Span\left\{\mat{1\\1\\0},\mat{-1\\0\\1}\right\}$.
				\end{quote}

				\item Find $\Rank(A)$.
				\begin{quote}
					Since the nullity of $A$ is 2 and $A$ is a $3\times 3$ matrix, the rank of $A$ must be 1.
				\end{quote}

				\item Find $\text{col}(A)$.
				\begin{quote}
					Since the rank of $A$ is 1, the dimension of the column space is 1.  Since $\mat{3\\2\\7}$ is
					contained in the column space, the column space must be $\Span\left\{\mat{3\\2\\7}\right\}$.
				\end{quote}

				\item Find $\text{row}(A)$.
				\begin{quote}
					Again, by the rank-nullity theorem, the row space of $A$ is one dimensional.  The
					row space is orthogonal to the null space.  By inspection we see that
					$\mat{1\\-1\\1}$ is orthogonal to the null space.  Thus, the row space is $\Span\left\{\mat{1\\-1\\1}\right\}$.
				\end{quote}

			\end{enumerate}
		
		\item Let
			\[
				\vec b_1=\mat{1\\1\\1}\qquad\vec b_2=\mat{1\\-1\\0}\qquad
				\vec b_3=\mat{-1\\0\\1}\qquad \vec c=\mat{1\\2\\3}
			\]
			and let $\mathcal B=\{\vec b_1,\vec b_2,\vec b_3\}$ and $\mathcal S=\{\vec e_1,\vec e_2,\vec e_3\}$.
			Suppose $T:\R^3\to\R^3$ is a linear transformation and $T(\vec b_1)=2\vec b_1$, 
			$T(\vec b_2)=3\vec b_2$, and $T(\vec b_3)=-\vec b_3$.
			\begin{enumerate}
				\item Compute $[\vec c]_{\mathcal B}$.
				\begin{quote}
					Since $\vec{c} = 2 \vec{b_1} + \vec{b_3}$, $[\vec c]_{\mathcal B} = \mat{2\\0\\1}$.
				\end{quote}

				\item Compute $[T\vec c]_{\mathcal B}$ and $[T\vec c]_{\mathcal S}$.
				\begin{quote}
					We have $T \vec{c} = T(2 \vec{b_1}) + T(\vec{b_3}) = 4 \vec{b_1} - \vec{b_3}$, so $[T\vec c]_{\mathcal B} = \mat{4\\0\\-1}$.  Converting this to the standard basis using given values of $\vec{b_1}$ and $\vec{b_3}$, we find $[T\vec c]_{\mathcal S} = \mat{5\\4\\3}$.
				\end{quote}

				\item Find a matrix for $T$ in the $\mathcal B$ basis  and a matrix for $T$ in
					the $\mathcal S$ basis. (In class, we might have said, ``A matrix for
					$T$ in the $\mathcal B$ coordinate system.''  This is another way of 
					saying, ``A matrix for $T$ in the $\mathcal B$ basis.'')
				\begin{quote}
					In the $\mathcal{B}$ basis, $T$ takes the basis vectors to $2 \vec{b_1}$, $3 \vec{b_2}$, and $- \vec{b_3}$, so its matrix is $\mat{2 & 0 & 0 \\ 0 & 3 & 0 \\ 0 & 0 & -1}$.
					
					In the standard basis, computing the matrix for $T$ is much harder.  Note that $\vec{e_1} = \frac{1}{3} \vec{b_1} + \frac{1}{3} \vec{b_2} - \frac{1}{3} \vec{b_3}$, so $T \vec{e_1} = \frac{2}{3} \vec{b_1} + \vec{b_2} + \frac{1}{3} \vec{b_3} = \mat{4/3\\-1/3\\5/3}$.
					
					Also, $\vec{e_2} = \frac{1}{3} \vec{b_1} - \frac{2}{3} \vec{b_2} - \frac{1}{3} \vec{b_3}$, so $T \vec{e_2} = \frac{2}{3} \vec{b_1} - 2 \vec{b_2} + \frac{1}{3} \vec{b_3} = \mat{-5/3 \\ 8/3 \\ 1}$.
					
					Finally, $\vec{e_3} = \frac{1}{3} \vec{b_1} + \frac{1}{3} \vec{b_2} + \frac{2}{3} \vec{b_3}$, so $T \vec{e_3} = \frac{2}{3} \vec{b_1} + \vec{b_2} - \frac{2}{3} \vec{b_3} = \mat{7/3 \\ -1/3 \\ 0}$.
					
					This means that the matrix for $T$ in the standard basis is $\mat{4/3 & -5/3 & 7/3 \\ -1/3 & 8/3 & -1/3 \\ 1 & 1 & 0}$.
					
					(There are other methods for computing the matrix for $T$ in the standard basis.  Try doing it using the $PAP^{-1}$ method from class, for instance.)
				\end{quote}

			\end{enumerate}
		
		\item Read section 3.1 and 3.2 in your textbook about computing the determinant of a matrix.
			Let $A=\mat{1&2&1\\1&1&1\\1&0&0}$ and $B=\mat{1&2&1\\1&1&1\\1&0&x}$.
		\begin{enumerate}
			\item Compute $\Det(A)$.
				\begin{quote}
					We'll expand along the bottom row, since it has two 0's in it.  The determinant is $1 \cdot (2 \cdot 1 - 1 \cdot 1) = 1$.
				\end{quote}

			\item Compute $\Det(B)$.  For what values of $x$ is $B$ not invertible?
				\begin{quote}
					We'll expand along the bottom row again.  The determinant is $1 \cdot (2 \cdot 1 - 1 \cdot 1) + x \cdot (1 \cdot 1 - 1 \cdot 2) = 1 - x$.
					
					$B$ is not invertible if the determinant is zero; that is, if $x=1$.  For all other values of $x$, $B$ is invertible.
				\end{quote}

		\end{enumerate}

		\item Let $A=\mat{1&2\\5&9}$.
		\begin{enumerate}
			\item Find an equation for the function $p(x)=\Det(A-xI)$ (this is called the
				\emph{characteristic polynomial} of $A$).
				\begin{quote}
					char$(A)=p(x)=x^2-10x-1$.
				\end{quote}

			\item For what values of $x$ is $A-xI$ non-invertible?
				\begin{quote}
					Exactly when $x=5\pm\sqrt{26}$
				\end{quote}

			\item Compute $p(A)$, the polynomial $p$ with the matrix $A$ plugged into it.  When you plug a matrix
				into a polynomial, replace any constant terms $k$ with the matrix $kI$.
				Can you guess
				why $p$ is called an \emph{annihilating} polynomial for $A$?
				\begin{quote}
					If we compute $A^2-10A-I$ we get the zero matrix.  $p(x)$ is called annihilating
					because when you ``plug in $A$'' you get the zero matrix.
				\end{quote}

		\end{enumerate}

		\item For each of the following, either give an example or a reason why it is impossible.
		\begin{enumerate}
			\item A linear transformation $T:\R^2\to\R^2$ that is invertible.
			\begin{quote}
				$\mat{x\\y}\mapsto\mat{x\\y}$.
			\end{quote}

			\item A linear transformation $T:\R^2\to\R^2$ that is not invertible.
			\begin{quote}
				$\mat{x\\y}\mapsto\mat{0\\0}$.
			\end{quote}

			\item A non-linear transformation $T:\R^2\to\R^2$.
			\begin{quote}
				$\mat{x\\y}\mapsto\mat{1\\1}$.
			\end{quote}

			\item A linear transformation $T:\R^2\to\R^2$ whose null space equals its range.
			\begin{quote}
				The linear transformation given by multiplication by the matrix
				$\mat{0&1\\0&0}$.
			\end{quote}

			\item A linear transformation $T:\R^2\to\R^2$ that when represented as a matrix in
			the standard basis has a column space equal to its row space.
			\begin{quote}
				The linear transformation given by multiplication by the matrix
				$\mat{1&1\\1&1}$.
			\end{quote}

			\item A linear transformation $T:\R^2\to\R^2$ so that $T^2$ is the identity, but $T$ is not invertible.
			\begin{quote}
				This is impossible.  If $T^2=id$, then $T=T^{-1}$, and so $T$ is invertible.
			\end{quote}

			\item A linear transformation $T:\R^2\to\R^2$ so that $T^3$ is the identity, but $T^2$ is not the identity.
			\begin{quote}
				Rotation counter clockwise by $120^{\circ}$.
			\end{quote}

			\item A linear transformation $T:\R^2\to\R^2$ with exactly \emph{one} eigenvector.
			\begin{quote}
				This is impossible.  If $T\vec v=\lambda \vec v$ and $T$ is linear, then $T(2\vec v)=2\lambda \vec v$
				and so $2\vec v$ is another eigenvector.
			\end{quote}

			\item A linear transformation $T:\R^2\to\R^2$ with exactly \emph{one} eigen direction (i.e., all eigenvectors lie on a single line).
			\begin{quote}
				The linear transformation given by multiplication by the matrix
				$\mat{1&1\\0&1}$.
			\end{quote}

			\item A linear transformation $T:\R^2\to\R^2$ with exactly \emph{two} eigen directions (i.e., all eigenvectors lie on one of two lines).
			\begin{quote}
				The linear transformation given by multiplication by the matrix
				$\mat{1&0\\0&2}$.
			\end{quote}

			\item A linear transformation $T:\R^2\to\R^2$ with \emph{infinitely many} eigen directions.
			\begin{quote}
				The linear transformation given by multiplication by the matrix
				$\mat{1&0\\0&1}$.
			\end{quote}

			\item A linear transformation $T:\R^2\to\R^2$ with no real eigenvectors.
			\begin{quote}
				Rotation counter clockwise by $90^{\circ}$.
			\end{quote}

			\item A linear transformation $T:\R^2\to\R^2$ with eigenvalues $3$ and $-2$.
			\begin{quote}
				The linear transformation given by multiplication by the matrix
				$\mat{2&0\\0&-3}$.
			\end{quote}

			\item A linear transformation $T:\R^2\to\R^2$ where $\mat{1\\1}$ is an eigenvector with eigenvalue $2$ and $\mat{2\\3}$
				is an eigenvector with eigenvalue $0$.
			\begin{quote}
				The linear transformation given by multiplication by the matrix
				$\mat{1&2\\1&3}\mat{2&0\\0&0}\mat{1&2\\1&3}^{-1}$.
			\end{quote}

		\end{enumerate}

	\item {\it This problem will not be on the exam but is included for you to stress-test your
		mathematical thinking.}
		
		Let $T:\R^n\to\R^n$. A subspace $X\subseteq \R^n$ is called \emph{invariant with respect to $T$} if $T(X)=X$.  That is,
		       $\{\vec v:\vec v=T\vec x\text{ for some }\vec x\in X\}=X$.  Note, this is \emph{different} than saying $T\vec x=\vec x$ for all $\vec x\in X$.
				\begin{quote}
					Note that $\{\vec{0}\}$ is an invariant subspace for any linear transformation.  In subsequent parts, we'll list only the other invariant subspaces.
				\end{quote}
		\begin{enumerate}
		       \item Describe all invariant subspaces of the linear transformation given by the matrix $\mat{1&0\\0&2}$.
			       \begin{quote}
				       The (nontrivial) invariant subspaces are $\R^2$, $\Span \{ \vec{e_1} \}$, and $\Span \{\vec{e_2}\}$.  (That is, $\R^2$ itself and the coordinate axes.)
				   \end{quote}
		       \item Describe all invariant subspaces of the linear transformation given by the matrix $\mat{2&0\\0&2}$.
			       \begin{quote}
				       Every subspace of $\R^2$ is an invariant subspace of this transformation!  This is because if $T$ is the transformation given, then $T\vec{x} = 2 \vec{x}$ for all $\vec{x} \in \R^2$.  $\R^2$ itself will be an invariant subspace, and any line through the origin will be ``stretched'' by this operation, making it also invariant.
				   \end{quote}
		       \item Describe all invariant subspaces of the shear given by the matrix $\mat{1&a\\0&1}$ where $a\neq 0$.
			       \begin{quote}
				       If $T$ is this transformation, then $T \mat{x\\y} = \mat{x + ay \\ y}$.  This is never in the same 1-dimensional subspace as $\mat{x\\y}$ unless $y=0$, so the only 1-dimensional invariant subspace is the $x$-axis, or $\Span \{\vec{e_1}\}$.  $\R^2$ is also an invariant subspace, since this matrix is invertible, as described below.
			       \end{quote}
			\item Your friend from another university proposes the following addition to the invertible matrix theorem:
				\begin{quote}
					An $n\times n$ matrix $A$ is invertible if and only if $\R^n$ is an invariant subspace of the transformation
					given by $A$.
				\end{quote}
				Is he right?  If so, prove it is correct.  If not, give a counterexample.
				
				\begin{quote}
					He is correct.  Let $\mathcal{A}$ be the corresponding transformation.  If $\R^n$ is an invariant subspace for $\mathcal{A}$, then the transformation must be onto (otherwise $\mathcal{A}(\R^n)$ would be smaller than $\R^n$).  If $\mathcal{A}$ is onto, then by the invertible matrix theorem, $A$ is invertible.
					
					On the other hand, if $A$ is invertible, then $\mathcal{A}$ is onto.  This means that $\mathcal{A}(\R^n)$ contains all of $\R^n$.  But $\mathcal{A}(\R^n)$ can't contain anything outside of $\R^n$, so it must equal $\R^n$, which means $\R^n$ is an invariant subspace for $\mathcal{A}$, as desired.
				\end{quote}
				
			\item Suppose $T:\R^2\to\R^2$ has $\{\vec 0\}$, $\left\{\mat{x\\y}:y=2x\right\}$, and $\R^2$ as its \emph{only}
				invariant subspaces.  Give an example of a vector $\vec v$ that is an eigenvector for $T$ and a vector $\vec w$
				that is \emph{not} an eigenvector for $T$.  Explain how you know.
				
				\begin{quote}
					The vector $\vec{v} = \mat{1\\2}$ must be an eigenvector for $T$.  $\vec{v}$ lies in a 1-dimensional invariant subspace of $T$, so $T \vec{v}$ must also be in this subspace.  But everything in this subspace is a scalar multiple of $\vec{v}$ (since the subspace is 1-dimensional), so $\vec{v}$ is an eigenvector.
					
					On the other hand, $\vec{w} = \mat{1\\0}$ cannot be an eigenvector for $T$.  If it is, with a nonzero eigenvalue, then $\Span \{\vec{w}\}$ would be an invariant subspace of $T$ (since $T$ would either ``stretch'' or ``shrink'' this line).  If it is with an eigenvalue of 0, then by the Invertible Matrix Theorem, $T$ is not invertible, and by our extension to the theorem, $\R^2$ is not an invariant subspace.  Neither of these is actually the case, so $\vec{w}$ can't be an eigenvalue for $T$.
				\end{quote}
		\end{enumerate}
		
		
	\end{enumerate}

\end{document}
