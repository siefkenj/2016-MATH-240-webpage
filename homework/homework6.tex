\documentclass[letter]{article}
\usepackage{amsmath}
\usepackage{amsfonts}
\usepackage{amssymb}
\usepackage{ifthen}
\usepackage{fancyhdr}
\usepackage{enumitem}

%%%
% Set up the margins to use a fairly large area of the page
%%%
\oddsidemargin=.2in
\evensidemargin=.2in
\textwidth=6in
\topmargin=0in
\textheight=9.0in
\parskip=.07in
\parindent=0in
\pagestyle{fancy}

%%%
% Set up the header
%%%
\newcommand{\setheader}[6]{
	\lhead{{\sc #1}\\{\sc #2} ({\small \it \today})}
	\rhead{
		{\bf #3} 
		\ifthenelse{\equal{#4}{}}{}{(#4)}\\
		{\bf #5} 
		\ifthenelse{\equal{#6}{}}{}{(#6)}%
	}
}

%%%
% Set up some shortcut commands
%%%
\newcommand{\R}{\mathbb{R}}
\newcommand{\N}{\mathbb{N}}
\newcommand{\Z}{\mathbb{Z}}
\newcommand{\Proj}{\mathrm{proj}}
\newcommand{\Perp}{\mathrm{perp}}
\newcommand{\proj}{\mathrm{proj}}
\newcommand{\Span}{\mathrm{span}}
\newcommand{\Null}{\mathrm{null}}
\newcommand{\Rank}{\mathrm{rank}}
\newcommand{\mat}[1]{\begin{bmatrix}#1\end{bmatrix}}
\renewcommand{\d}{\mathrm{d}}

%%%
% This is where the body of the document goes
%%%
\begin{document}
	\setheader{Math 240}{Homework 6}{Due: Thursday, November 17}{}{}{}
	\emph{We are nearly experts at typing by now.  Though you are not required
	to type your homework, it's strongly encouraged.  You can even download
	the \texttt{tex} file for this homework and type your answers below
	each problem.  Using the} \verb!\begin{quote}! \emph{and} \verb!\end{quote}! \emph{environment
		will indent anything you type inbetween.  Perfect for typing answers!}
	
	\begin{enumerate}
		
		%\item Let $\vec{u} = \mat{1\\1\\1}$, $\vec{v} = \mat{1\\-1\\-1}$, $\vec{w} = \mat{-1\\1\\-1}$, and $\vec{x} = \mat{2\\0\\0}$.
		%\begin{enumerate}
		%	\item Find a linear transformation $T_1$ such that $T_1(\vec{e_1}) = \mat{1\\2\\3}$, $T_1(\vec{e_2}) = \mat{4\\5\\6}$, and $T_1(\vec{e_3}) = \mat{7\\8\\9}$.
		%	\item Find a linear transformation $T_2$ such that $T_2(\vec{u}) = \mat{1\\2\\3}$, $T_2(\vec{v}) = \mat{4\\5\\6}$, and $T_2(\vec{w}) = \mat{7\\8\\9}$.
		%	\item Is there a linear transformation $T_3$ such that $T_3(\vec{u}) = \mat{1\\2\\3}$, $T_3(\vec{v}) = \mat{4\\5\\6}$, and $T_3(\vec{x}) = \mat{7\\8\\9}$?  Explain why or why not.
		%\end{enumerate}
		%You may give your answers in the form of a matrix.
		
		\item Let $A = \mat{1 & 0 & 1 & 1 & 3 \\ 1 & 2 & 3 & -1 & -1 \\ 3 & 1 & 4 & 2 & 7}$.  Let $\mathcal{C}$, $\mathcal{R}$, and $\mathcal{N}$ be the column, row, and null spaces of $A$, respectively.
		\begin{enumerate}
			\item Find a basis for $\mathcal{C}$
			\item Find a basis for $\mathcal{R}$.
			\item Find a basis for $\mathcal{N}$.
			\item Find the dot product of each of your basis vectors for $\mathcal{R}$ with each of your basis vectors for $\mathcal{N}$.
			What can you say geometrically about $\mathcal{R}$ and $\mathcal{N}$?
		\end{enumerate}
		
		\item Let $L$ be the line $x=y=z$ in $\R^3$.
		\begin{enumerate}
			\item Find a $3\times 3$ matrix $B_1$ whose column space is the $xy$-plane and whose null space is $L$.
			\item Find a $3\times 3$ matrix $B_2$ whose column space is the $xz$-plane and whose null space is $L$.
			\item Find the row space of $B_1$ and the row space of $B_2$.  How do they compare?  Explain.
		\end{enumerate}
		
		\item A linear transformation $T$ has the following effects:
		\begin{itemize}
			\item Along the line $y = x$, it ``shrinks'' everything by a factor of 2---points along this line move halfway to the origin.
			\item Along the line $y=-3x$, it reflects everything over the origin---if $\vec{v}$ is the position vector of a point on this line, then it moves to $-\vec{v}$.
		\end{itemize}
		\begin{enumerate}
			\item Find a matrix $A$ such that $T(\vec{v}) = A\vec{v}$ for all $\vec{v} \in \R^2$.
			\item Find $A^{-1}$.
			\item What are the ``stretch factors'' and ``stretch directions'' of the transformation given by $A^{-1}$?
		\end{enumerate}

		%\item Let $T:\R^n\to\R^n$. A subspace $X\subseteq \R^n$ is called \emph{invariant with respect to $T$} if $TX=X$.  That is,
			%$\{\vec v:\vec v=T\vec x\text{ for some }\vec x\in X\}=X$.  Note, this is \emph{different} than saying $T\vec x=\vec x$ for all $\vec x\in X$.
		%\begin{enumerate}
			%\item Describe all invariant subspaces of the linear transformation given by the matrix $\mat{1&0\\0&2}$.
			%\item Describe all invariant subspaces of the linear transformation given by the matrix $\mat{2&0\\0&2}$.
			%\item Describe all invariant subspaces of the sheer given by the matrix $\mat{1&a\\0&1}$ where $a\neq 0$.
			%\item Explain how invariant subspaces relate to eigenvectors.
		%\end{enumerate}
		
		\item Recall the ``italicizing $N$'' matrix that you found in class: $A = \mat{1 & \frac{1}{3} \\ 0 & \frac{4}{3}}$.
		Find all eigenvalues (``stretch factors'') and eigenvectors (``stretch directions'') of $A$.
		
		%\item Give an example of a 2-by-2 matrix with fewer than two ``stretch directions''.  (That is, find a matrix with fewer than two linearly indepedent eigenvectors.)
			
		
		
			
		
	\end{enumerate}

\end{document}
