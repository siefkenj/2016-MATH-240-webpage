\documentclass[letter]{article}
\usepackage{amsmath}
\usepackage{amsfonts}
\usepackage{amssymb}
\usepackage{ifthen}
\usepackage{fancyhdr}

\usepackage{enumitem}

%%%
% Set up the margins to use a fairly large area of the page
%%%
\oddsidemargin=.2in
\evensidemargin=.2in
\textwidth=6in
\topmargin=0in
\textheight=9.0in
\parskip=.07in
\parindent=0in
\pagestyle{fancy}

%%%
% Set up the header
%%%
\newcommand{\setheader}[6]{
	\lhead{{\sc #1}\\{\sc #2} ({\small \it \today})}
	\rhead{
		{\bf #3} 
		\ifthenelse{\equal{#4}{}}{}{(#4)}\\
		{\bf #5} 
		\ifthenelse{\equal{#6}{}}{}{(#6)}%
	}
}

%%%
% Set up some shortcut commands
%%%
\newcommand{\R}{\mathbb{R}}
\newcommand{\N}{\mathbb{N}}
\newcommand{\Z}{\mathbb{Z}}
\newcommand{\Proj}{\mathrm{proj}}
\newcommand{\Perp}{\mathrm{perp}}
\newcommand{\proj}{\mathrm{proj}}
\newcommand{\Span}{\mathrm{span}}
\newcommand{\Null}{\mathrm{null}}
\newcommand{\Rank}{\mathrm{rank}}
\newcommand{\mat}[1]{\begin{bmatrix}#1\end{bmatrix}}
\newcommand{\sbv}[1]{\ensuremath{\mathbf{e}_{#1}}}

%%%
% This is where the body of the document goes
%%%
\begin{document}
	\setheader{Math 240}{Homework 2}{Due: Thursday, October 6th}{}{}{}
	Over the weekend, \textbf{read sections 1.1, 1.2, 1.4, and 2.1} in the textbook.  (Out of 2.1, you just need to read how to multiply matrices.)  You should complete the reading before class on Wednesday, because we'll use the material right away on Wednesday.  Some homework problems require using what you learn in these sections; they are marked with the appropriate section number.
	\begin{enumerate}
		
		\item (sec. 1.2) Find the reduced row echelon form of each of the following matrices:
		
		\begin{tabular}{cc}
			(a) $\mat{1 & 4 & 1 & -2 \\ 3 & 11 & -4 & 0 \\ 2 & 2 & 1 & 1}$
			&
			(b) $\mat{1 & 6 & 4 & 2 \\ 1 & 3 & 1 & 4 \\ 1 & 8 & 6 & -1}$
		\end{tabular}
		
		\item (sec. 2.1) Compute the following matrix products:
		
		\begin{tabular}{cc}
			(a) $\mat{2 & 1 \\ 1 & -1} \mat{1 & 3 \\ 2 & 2}$
			&
			(b) $\mat{1 & 1 & 1 & 1 \\ 2 & 1 & 1 & -1} \mat{2 & 3 & 4 \\ 3 & 4 & 5 \\ 4 & 5 & 6 \\ -1 & -1 & -1}$
		\end{tabular}
		
		\item Write each of the following vectors as a linear combination of $\vec{u} = \mat{2\\-1\\-1}$ and $\vec{v} = \mat{1\\3\\1}$, or explain why it is impossible.
		
		\begin{tabular}{ccc}
			(a) $\mat{1\\10\\4}$
			&
			(b) $\mat{6\\-4\\2}$
			&
			(c) $\mat{0\\0}$
		\end{tabular}
		
		\item Consider the same vectors $\vec{u}$ and $\vec{v}$ from the previous problem.
		\begin{enumerate}
			\item What sort of geometric object is $\Span \{\vec{u},\vec{v}\} $?  Explain how you know.
			\item Give one example of a vector in $\Span \{\vec{u},\vec{v}\} $, and one example of a vector not in $\Span \{\vec{u},\vec{v}\} $.  Explain.
			\item Is it correct to say that $\Span \{\vec{u},\vec{v}\} = \R^2$?  Why or why not?
		\end{enumerate}
		
		\item Suppose $\vec{u}$, $\vec{v}$, and $\vec{w}$ are nonzero vectors in $\R^3$, and $\vec{u}$ is a linear combination of $\vec{v}$ and $\vec{w}$.
		\begin{enumerate}
			\item Can you determine from this information whether or not the set $\{\vec{u}, \vec{v},\vec{w} \}$ is linearly independent?  Explain.
			\item Can you determine from this information whether or not $\vec{v}$ is a linear combination of $\vec{u}$ and $\vec{w}$?  Explain.
		\end{enumerate}
		
		\item Express each of the following statements in formal mathematical language.  Note that there is more than one way to do so in each case, and your answers may include mathematical vocabulary as wells as expressions (sets, formulas, equations, etc.).  $\vec{u}$, $\vec{v}$, and $\vec{w}$ represent fixed vectors in $\R^2$.
		\begin{enumerate}
			\item By traveling in the direction of $\vec{u}$ and then the direction of $\vec{v}$, you can reach any point in the $xy$-plane.
			\item By traveling in the direction of $\vec{w}$ and then the direction $\vec{v}$, you cannot reach the point $(2,18)$.
			\item Starting anywhere, you can make a ``loop'' traveling only in the directions of $\vec{u}$, $\vec{v}$, and $\vec{w}$, and come back to where you started.
		\end{enumerate}

		
	\end{enumerate}
	
	\textbf{Don't forget to also complete PAR2, this week's Peer-Assisted Reflection problems.}
	




\end{document}
