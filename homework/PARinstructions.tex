\documentclass[letter]{article}
\usepackage{amsmath}
\usepackage{amsfonts}
\usepackage{amssymb}
\usepackage{ifthen}
\usepackage{fancyhdr}

%%%
% Set up the margins to use a fairly large area of the page
%%%
\oddsidemargin=.2in
\evensidemargin=.2in
\textwidth=6in
\topmargin=0in
\textheight=9.0in
\parskip=.07in
\parindent=0in
\pagestyle{fancy}

%%%
% Set up the header
%%%
\newcommand{\setheader}[6]{
	\lhead{{\sc #1}\\{\sc #2} ({\small \it \today})}
	\rhead{
		{\bf #3} 
		\ifthenelse{\equal{#4}{}}{}{(#4)}\\
		{\bf #5} 
		\ifthenelse{\equal{#6}{}}{}{(#6)}%
	}
}

%%%
% Set up some shortcut commands
%%%
\newcommand{\R}{\mathbb{R}}
\newcommand{\N}{\mathbb{N}}
\newcommand{\Z}{\mathbb{Z}}
\newcommand{\Proj}{\mathrm{proj}}
\newcommand{\Perp}{\mathrm{perp}}
\newcommand{\proj}{\mathrm{proj}}
\newcommand{\Span}{\mathrm{span}}
\newcommand{\Null}{\mathrm{null}}
\newcommand{\Rank}{\mathrm{rank}}
\newcommand{\mat}[1]{\begin{bmatrix}#1\end{bmatrix}}

%%%
% This is where the body of the document goes
%%%
\begin{document}
	\setheader{Math 240}{PAR Instructions}{$\phantom{x}$}{}{}{}
	\rfoot{\footnotesize\it Based on work of Daniel Reinholz }
	In this class, we will be spending a lot of time tackling hard concepts
	and developing deep understanding.  One of the best ways to refine
	your thinking and to get a boost when you're stuck is to explain
	your reasoning.  In this class, we will use an exercise called Peer-Assisted
	Reflection (PAR) to give you practice at analysing your own thoughts
	as well as understanding how others think.  Research has shown
	as much as a 20\% improvement in student learning by using PAR.  We're
	including PAR in this course because we want
	to provide you with the best learning techniques we know of!

	\begin{enumerate}
		\item {\bf PAR Basics}
			\begin{enumerate}
				\item Roughly one homework problem a week will be designated as a PAR problem.
					You work on this problem alone or with peers and write up your solution.
					You will then complete the self-reflection sheet for that problem.
				\item In discussion section, you will sit in a random seat.
					This is so you get exposure to a large sample of your peers' work, not just your friends!
				\item You will exchange write-ups with your neighbor, read it carefully, and spend some time providing
					them written feedback.  
					It is important you provide written feedback before you start talking with your partner.
					When you are reading another
					student's work, you are focusing on what they've written
					and how they expressed their thoughts, not the problem itself.
					(You've already spend a lot of time thinking about the problem!)
				\item Conference with your neighbor and explain
					your feedback in words.
					Providing feedback is not an option. Even the best solutions 
					can be made better. Even if you did not complete the problem,
					you can still engage in written feedback and conferencing.
				\item At home, you revise your PAR question and turn in 
					your PAR work the next day of class.  Your PAR work includes
					(i) your draft with comments, (ii) your self-reflection, and
					(iii) a \emph{typed} final writeup of your PAR question.
			\end{enumerate}
		\item {\bf Frequently Asked Questions}
			\begin{enumerate}
				\item {\it What if I miss a PAR day? Can I still get credit?}
				Yes. You can do the peer
				conference with one of your classmates outside of class and turn the PAR problem in as normal.
				\item {\it If I didn't solve a PAR problem, do I still participate?}  Yes!  Even
					if you haven't figured out the problem, you can give valuable feedback and
					pick up some new ideas from your peers.
			\end{enumerate}
	\end{enumerate}

	\section*{Being a Critical Friend}

	Peer-Assisted Reflection (PAR) will help you improve your communication skills and understanding 
	of this math course. 
	Writing for someone who does not already understand your problem is a skill that requires
	practice. We will improve our skills by acting as ``critical friends'' to each other. 
	Critical friends take the time to fully understand each other's work, 
	finding the flaws and weaknesses so that they can be improved.
	Critical friends don't just say ``everything looks good,'' because it provides no room for growth.

	\subsection*{Reading Peer's Solutions}
	\begin{itemize}
		\item Try to understand the solution based only on what is
		written (without thinking of your own solution).  Note any
		gaps in logic you have to fill in yourself. Tell your
		partner---these need to be revised.  
		
		\item Was the solution
		justified? What evidence would you need to be sure the
		solution is correct?  
	
		\item Note any calculation or
		procedural errors you find. Don't just gloss over
		calculations, but work them out in your head to see
		if your partner's work makes sense.  
	
		\item After you've fully considered what your partner wrote,
		compare their solution to your own. Did you
		get the same result? If you used different approaches,
		were they consistent or did your solutions contradict
		one another? The more you connect the two solutions, the
		more you will deepen your understanding.
	\end{itemize}
	\subsection*{Peer Conferencing}
		A few simple guidelines will help you have more meaningful conversations with your partner.
	\begin{itemize}
		\item Focus on the mathematical aspects of the solution
		to the problem \emph{\textbf{and}} how they were communicated.

		\item Ask questions. Encourage your partner to ask
		questions. Discussing ideas will help you learn.

		\item Demand meaningful feedback. If your partner only
		says ``everything looks good,'' you learn nothing.

		\item Practice revisions to your explanations (verbally)
		before writing them. This is a unique opportunity to
		get instant feedback on your communication. Use it!

		\item If both you and your partner are unsure about the
		solution, try to figure it out together. Talk through your
		reasoning, where it got you, and where it got you stuck.
	\end{itemize}

	\subsection*{Meaningful Feedback}
		Use the self-reflection questions as a guide for areas you might give feedback to your partner on.
	\begin{itemize}
		\item Be specific. Saying ``your explanation was
			unclear'' is not nearly as useful as saying \emph{why} it
		was unclear (e.g., ``you used the term asymptote
		incorrectly.'') If you say \emph{why} you think something,
		your partner can evaluate your feedback, regardless of
		your confidence.

		\item Be constructive. As much as possible, don't just
		say that something needs to be improved, but suggest how
		it could be improved (e.g., ``asymptotes \emph{can} be touched
		by the function; you know you have an asymptote because
		you have a certain limit'').

		\item Try to help your partner learn. If the solution
		seems to indicate misunderstanding of a concept, spend
		time talking about that concept until both you and your
		partner are clear on it. Explaining and discussing
		concepts with your partner will help you understand
		them better.
	\end{itemize}

	\subsection*{Not Very Meaningful Feedback}
		Here are a number of examples of feedback that are not very helpful for you partner 
		(\emph{explanations why in
		italics}). Please refrain from offering these types of feedback.

		{\bf Communication:}
	\begin{itemize}
		\item Everything looked good. (\emph{Even the best
		solutions can be improved. Put more effort into
		thinking how
			your partner could do so.})

		\item Your solution was explained well. (\emph{This
		feedback not only doesn't say what about it was explained
			well, but it also doesn't help your partner
			improve.})

		\item Your solution could use a little bit more
		explanation. (\emph{Tell your partner exactly what
		needs more explanation and try to suggest how they could
		improve their explanation. Remember, simply adding \textbf{more}
		isn't \textbf{always} better.})

		\item Your explanations were a little unclear. (\emph{You need
		to tell your partner what exactly was unclear, and try
		to suggest how they could improve it. Try focusing on
		what you couldn't understand or where you got lost.})

		\item Show a little bit more work. (\emph{Tell your partner
		exactly where you got confused. You need to be more
		specific with your feedback.})
	\end{itemize}

	{\bf Correctness:}
	\begin{itemize}
		\item I found an error in part (e). (\emph{Tell your partner what the error was and why it was incorrect.})
		\item Your answer to part (a) needs improvement. (\emph{Tell your partner what was wrong and how to improve
			it.})
		\item I think your solution is wrong. (\emph{Tell your partner what the error was and why it was incorrect.})
		\item Be sure to finish the problem. (\emph{Your partner probably didn't finish the problem because they didn't
			know how to. Instead of telling them the obvious, try to help them figure it out.})
		\item Everything looks good. (\emph{What evidence do you have for saying this? })
	\end{itemize}

	\subsection*{Strategies for Facilitating Peer-Conferences}
	Sometimes it may feel like you don't know what to say or how to get better feedback from your peer
	conferences. If your partner simply says ``it looks good,''
	it does nothing to help you improve. PAR is a
	unique opportunity to get feedback and talk about the mathematics in a problem, so make sure that you
	take advantage of it!
	Here are some suggestions for things you might say or ask in order to have a more productive 
	conversation. One key strategy is asking questions about specific aspects of your solution or your partner's
	solution. Here are some suggestions:
	\begin{itemize}
		\item I was struggling with how to communicate X, do you have any suggestions?
		\item Would it be all right if I practiced my explanation with you before I revise my solution?
		\item Did you check all of my calculations, and did you find any errors?
		\item I noticed that we did the problem differently. Can we look together and make sure our solutions are
		consistent?
		\item I was unsure about (concept X, or how to do X). Can we talk more about it?
		\item In your solution, I noticed that you did X. Can you explain why?
	\end{itemize}
\end{document}
