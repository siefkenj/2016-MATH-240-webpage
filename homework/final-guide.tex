\documentclass[10pt]{article}
\usepackage{amsmath, amsfonts, amsthm}
\usepackage{fancyhdr,parskip}
\usepackage{fullpage}
\usepackage[answerdelayed]{exercise}


\setlength{\headheight}{15.2pt}
\setlength{\headsep}{20pt}
\pagestyle{fancyplain}

\renewcommand{\ExerciseHeader}{\textbf{Problem \ExerciseHeaderNB .}}
\renewcommand{\AnswerHeader}{\textbf{Problem \ExerciseHeaderNB .}}

\newcommand{\be}[1]{\begin{Exercise}[label=#1]}
\newcommand{\ee}{\end{Exercise}}
\newcommand{\ba}[1]{\begin{Answer}[ref=#1]}
\newcommand{\ea}{\end{Answer}}
\newcommand{\ds}{\displaystyle}
\newcommand{\is}{\hspace{2pt}}
\newcommand{\dx}{\is dx}

\newcommand{\Z}{\mathbb{Z}}
\newcommand{\Q}{\mathbb{Q}}
\newcommand{\R}{\mathbb{R}}
\newcommand{\C}{\mathbb{C}}
\newcommand{\F}{\mathbb{F}}


\fancyhead[L]{Skills Check 1 Guide}
\fancyhead[C]{}
\fancyhead[R]{Math 240, Fall 2016}


\begin{document}
	
This is the topic list for the final skills check.  It includes ALL topics from the first two skills checks, plus some new material.  New material is in \textbf{bold}.

You should be able to give a mathematically precise definition of the following:

\begin{itemize}
	\item linear combination
	\item linearly independent/dependent
	\item span
	\item consistent/inconsistent
	\item free variable
	\item trivial/nontrivial linear combination
	\item \ldots and any other terminology or notation we've discussed in class.
	\item Linear transformation
	\item Inverse of a linear transformation
	\item Inverse of a matrix
	\item Elementary matrix
	\item Rank of a matrix
	\item Subspace
	\item Basis
	\item Dimension
	\item Range
	\item Null space
	\item Row space
	\item Column space
	\item \textbf{Eigenvalue}
	\item \textbf{Eigenvector}
	\item \textbf{Eigenspace}
	\item \textbf{Characteristic polynomial}
	\item \textbf{Diagonalization/diagonalizable}
\end{itemize}

You should be able to state the following theorems:
\begin{itemize}
	\item The Rank-Nullity Theorem
	\item The Invertible Matrix Theorem (or most of it)
\end{itemize}

You should be able to do the following computations:
\begin{itemize}
	\item Give mathematically precise definitions for all terms used in class, particularly including the ones used later on this list.
	\item Do vector arithmetic: addition, subtraction, scalar multiplication, and dot product.
	\item Multiply matrices.
	\item Express a vector as a linear combination of other vectors.
	\item Determine whether or not a set of vectors is linearly independent.
	\item Determine whether one vector is in the span of others.
	\item Find the span of a set of vectors.
	\item Put a matrix into reduced row echelon form, and recognize whether or not it is.
	\item Convert a system of linear equations into augmented-matrix form and matrix-equation (Av=b) form, and vice versa.
	\item Find all solutions to a system of linear equations.
	\item Given a linear transformation $T$, compute $T(\vec{v})$.
	\item Given a linear transformation $T$, find a matrix for $T$.
	\item Find the inverse of a matrix.
	\item Use several different methods to determine whether or not a matrix is invertible.
	\item Determine whether or not a transformation is linear.
	\item Determine whether or not something is a subspace.
	\item Given a subspace, find a basis for it and compute its dimension.
	\item Describe the range and null space of a linear transformation.
	\item Give a basis for the range and null space of a linear transformation.
	\item Describe the row space, column space, and null space of a matrix.
	\item Give a basis for the row space, column space, and null space of a matrix.
	\item \textbf{Given a vector in one basis, convert it to another basis.}
	\item \textbf{Given a linear transformation expressed as a matrix in one basis, express it as a matrix in another basis.}
	\item \textbf{Given two bases, find a change-of-basis matrix to convert between them.}
	\item \textbf{Compute the eigenvalues of a linear transformation, and their corresponding eigenvectors.}
	\item \textbf{Given a linear transformation and one of its eigenvectors, find the corresponding eigenvalue.}
	\item \textbf{Given a linear transformation and one of its eigenvalues, find a basis for the corresponding eigenspace.}
	\item \textbf{Given the eigenvalues and eigenvectors of a linear transformation, find a matrix for it.}
	\item \textbf{Given a linear transformation (or a matrix for it), find its characteristic polynomial.}
\end{itemize}

Also remember that problems from HW7 (or close variations on them) may appear on the skills check!



\end{document}
