\documentclass[letter]{article}
\usepackage{amsmath}
\usepackage{amsfonts}
\usepackage{amssymb}
\usepackage{amsthm}
\usepackage{mdframed}
\usepackage{ifthen}
\usepackage{fancyhdr}
\usepackage[usenames,dvipsnames,svgnames,table]{xcolor}
\usepackage{tikz}
\usepackage{changepage}
\graphicspath{{images/}}
\usepackage[hidelinks]{hyperref}
\usepackage{mathrsfs}

%%%
% Set up the margins to use a fairly large area of the page
%%%
\oddsidemargin=.2in
\evensidemargin=.2in
\textwidth=6in
\topmargin=0in
\parskip=.07in
\parindent=0in
\pagestyle{fancy}

\expandafter\def\expandafter\quote\expandafter{\quote\sf\color{DarkGreen}}

%%%
% Set up the header
%%%
\newcommand{\setheader}[7]{
	\lhead{{\sc #1}\\{\sc #2} %({\small \it \today})
	}
	\rhead{
		{\bf #3} 
		\ifthenelse{\equal{#4}{}}{}{(#4)}\\
		{\bf #5} 
		\ifthenelse{\equal{#6}{}}{#7}{(#6)}%
	}
}

%%%
% Set up closure command
%%%
\newcommand{\overbar}[1]{\mkern 1.5mu\overline{\mkern-1.5mu#1\mkern-1.5mu}\mkern 1.5mu}
%%%
% Set up some shortcut commands
%%%
\newcommand{\R}{\mathbb{R}}
\newcommand{\N}{\mathbb{N}}
\newcommand{\Z}{\mathbb{Z}}
\newcommand{\Q}{\mathbb{Q}}
\newcommand{\Proj}{\mathrm{proj}}
\newcommand{\Perp}{\mathrm{perp}}
\newcommand{\Span}{\mathrm{span}}
\newcommand{\Null}{\mathrm{null}}
\newcommand{\Rank}{\mathrm{rank}}
\newcommand{\Range}{\mathrm{range}}
\newcommand{\Col}{\mathrm{col}}
\newcommand{\Row}{\mathrm{row}}
\newcommand{\Det}{\mathrm{det}}
\newcommand{\mat}[1]{\begin{bmatrix}#1\end{bmatrix}}
\newcommand{\Rref}{\mathrm{rref}}
\renewcommand{\d}{\mathrm{d}}
\newcommand{\abs}[1]{\left\vert #1 \right\vert}
\newcommand{\norm}[1]{\left\| #1 \right\|}
\newcommand{\inner}[2]{\langle #1, #2 \rangle}
\newcommand{\RI}{\mathscr{R}}
\newcommand{\RSI}[1]{\mathscr{R}(#1)}

\DeclareMathOperator{\arccot}{arccot}
\DeclareMathOperator{\diam}{diam}

\def\upint{\mathchoice%
    {\mkern13mu\overline{\vphantom{\intop}\mkern7mu}\mkern-20mu}%
    {\mkern7mu\overline{\vphantom{\intop}\mkern7mu}\mkern-14mu}%
    {\mkern7mu\overline{\vphantom{\intop}\mkern7mu}\mkern-14mu}%
    {\mkern7mu\overline{\vphantom{\intop}\mkern7mu}\mkern-14mu}%
  \int}
\def\lowint{\mkern3mu\underline{\vphantom{\intop}\mkern7mu}\mkern-10mu\int}

\newenvironment{answer}{
	\begin{adjustwidth}{8mm}{} \vspace{2mm}}{\end{adjustwidth} \vspace{2mm}
}

\theoremstyle{plain}
\newtheorem*{theorem}{Theorem}
\newtheorem{lemma}{Lemma}
\newtheorem*{prop}{Proposition}

\theoremstyle{definition}
\newtheorem*{definition}{Definition}

\theoremstyle{remark}
\newtheorem*{claim}{Claim}

%%%
% This is where the body of the document goes
%%%
\begin{document}
	\setheader{Math 240}{Homework 4 Solutions}{Due: Thursday, October 27}{}{}{}{}
	\begin{enumerate}
		
		\item \textbf{Reading Assignment:} Read and practice the algorithm for finding the inverse of a matrix, found at the end of section 2.2 of the textbook.
		
		\item For each of the following transformations, find a matrix corresponding to the transformation.
		
		\begin{enumerate}
			\item $\mathcal{A}$: Rotate $90^\circ$ counterclockwise around the origin (in $\R^2$).
			\begin{answer}
			Let $M_A$ be the corresponding matrix for $A$. Then
			\[
			M_A\mat{1\\0} = \mat{0\\1} \text{ and } M_A \mat{0\\1} = \mat{-1\\0} \implies M_A = \mat{0 & -1 \\ 1 & 0}.
			\]
			We find the rest of the matrices in this problem similarly. 
			\end{answer}
			\item $\mathcal{B}$: Send every vector in $\R^2$ to the zero vector.
			\begin{answer}
				$\displaystyle M_B = \mat{0&0\\0&0}$.
			\end{answer}
			\item $\mathcal{C}$: Project (in $\R^3$) onto the $yz$-plane.
			\begin{answer}
				$\displaystyle M_C = \mat{0&0&0\\0&1&0\\0&0&1}$.
			\end{answer}
			\item $\mathcal{D}$: Project (in $\R^3$) onto the $x$-axis.
			\begin{answer}
				$\displaystyle M_D = \mat{1&0&0\\0&0&0\\0&0&0}$.
			\end{answer}
			\item $\mathcal{E}$: Reflect (in $\R^3$) across the $xy$-plane.
			\begin{answer}
				$\displaystyle M_E = \mat{1&0&0\\0&1&0\\0&0&-1}$.
			\end{answer}
			\item $\mathcal{F}$: Stretch by a factor of 2 in the $y$-direction (in $\R^2$).
			\begin{answer}
				$\displaystyle M_F = \mat{0&0\\0&0}$.
			\end{answer}
			\item $\mathcal{G}$: Stretch by a factor of 2 in the $y$-direction (in $\R^3$).
			\begin{answer}
				$\displaystyle M_G = \mat{1&0&0\\0&2&0\\0&0&1}$.
			\end{answer}
			\item $\mathcal{H}$: Rotate $90^\circ$ counterclockwise (as viewed looking ``down'' from the positive $y$-axis towards the origin) around the $y$-axis (in $\R^3$).
			\begin{answer}
				$\displaystyle M_H = \mat{0&0&1\\0&1&0\\-1&0&0}$.
			\end{answer}
		\end{enumerate}
		
		
		\item \begin{enumerate}
			\item For each transformation in problem 2, explain geometrically whether or not the transformation is invertible.
			\begin{answer}
				\begin{itemize}
					\item Since we can rotate $90^\circ$ clockwise to undo the effect of $\mathcal{A}$, $\mathcal{A}$ must be invertible. 
					\item Since multiple vectors in $\R^2$ are transformed to $\vec{0}$, it is impossible to invert the transformation $\mathcal{B}$. 
					\item Similarly, $\mathcal{C}$ and $\mathcal{D}$ are not invertible, while $\mathcal{E}, \mathcal{F}, \mathcal{G}, \mathcal{H}$ are invertible. 
				\end{itemize}
			\end{answer}
			\item Which of the matrices you generated in problem 2 are invertible?  (You may wish to use the algorithm you read about.)
			\begin{answer}
				By using the algorithm for finding the inverse of a matrix, we see that matrices $M_A, M_E, M_F, M_G, M_H$ are invertible, while $M_B, M_C, M_D$ are not. 
			\end{answer}
			\item Suppose $f$ is a transformation that is represented by the matrix $M$.
				Is it possible for $f$ to be invertible and $M$ not invertible?  Is it possible
				for $M$ to be invertible and $f$ not be invertible? Explain your reasoning.
			\begin{answer}
				Assume that $f$ is a transformation on $\R^n$ represented by the matrix $M$. Hence, for any vector $\vec{v} \in \R^n$, $f(\vec{v}) = M\vec{v}$. 
				\begin{itemize}
					\item If $M$ is invertible, then there must exist some matrix $M^{-1}$ so that $MM^{-1} = M^{-1}M = I$. Let $g$ be a transformation represented by the matrix $M^{-1}$. Then, for all $\vec{v} \in \R^n$, we have that $g(f(\vec{v})) = M^{-1} M \vec{v} = I \vec{v} = \vec{v}$ and also that $f(g(\vec{v})) = MM^{-1}\vec{v} = I \vec{v} = \vec{v}$. By definition, therefore, $g$ is the inverse of $f$, so $f$ must be invertible. 
					\item If $f$ is invertible, then the mapping $f: \R^n \to \R^n$ is one-to-one. The same is true for the mapping $M: \R^n \to \R^n$. Therefore, $\vec{x} = \vec{0}$ is the unique solution for the equation $M\vec{x} = \vec{0}$. By the Inverse Matrix Theorem, $M$ is invertible. 
				\end{itemize}
			\end{answer}
		\end{enumerate}
		
		
		\item Each of the following matrices gives a transformation of $\R^2$ or $\R^3$.  
			Describe the effect of that transformation in terms of the transformations from problem 2.
			(Hint: you may need to combine two, three, or more transformations from problem 2 for some
			examples.)
			\[
				X=\mat{0 & -1 \\ 2 & 0}
				\qquad \qquad
				Y=\mat{0 & 0 & 0\\ 0 & 1 & 0 \\ 0 & 0 & -1}
			\]
			\[
				Z=\mat{0 & 0 & 1 \\ 0 & 2 & 0 \\ -1 & 0 & 0}
				\qquad\qquad
				W=\mat{0 & 0 & 1 \\ 0 & 0 & 0 \\ 0 & 0 & 0}
			\]
		\begin{answer}
			\begin{itemize}
				\item
				\[
				X = \mat{0 & -1 \\ 2 & 0}
				\]
				Observe that
				\[
				\mat{0 & -1 \\ 2 & 0}\mat{x\\y} = \mat{-y\\2x}.
				\]
				$X$ rotates $90^\circ$ counterclockwise and then stretches by a factor of $2$ in the $y$-direction (in $\R^2$). Therefore, $X = \mathcal{F} \circ \mathcal{A}$. We check
				\[
				M_F M_A = \mat{1&0\\0&2}\mat{0&-1\\1&0} = \mat{0&-1\\2&0} = M_X.
				\]
				\item Similarly, we can deduce
				\begin{itemize}
					\item $Y = \mathcal{C} \circ \mathcal{E}$ since
					\[
					M_C M_E = \mat{0&0&0\\0&1&0\\0&0&1}\mat{1&0&0\\0&1&0\\0&0&-1} = \mat{0&0&0\\0&1&0\\0&0&-1}.
					\]
					\item $Z = \mathcal{G} \circ \mathcal{H}$ since
					\[
					M_G M_H = \mat{1&0&0\\0&2&0\\0&0&1}\mat{0&0&1\\0&1&0\\-1&0&0} = \mat{0&0&1\\0&2&0\\-1&0&0} = M_Z.
					\]
					\item $W = \mathcal{D} \circ \mathcal{H}$ since
					\[
					M_D M_H = \mat{1&0&0\\0&0&0\\0&0&0}\mat{0&0&1\\0&1&0\\-1&0&0} = \mat{0&0&1\\0&0&0\\0&0&0} = M_W.
					\]
				\end{itemize}
			\end{itemize}
		\end{answer}
		\item Use the definition of a linear transformation
			to show that if $T$ is a linear transformation, then $T(\vec{0}) = \vec{0}$.  
			Do not appeal to the matrix representation of $T$.
		\begin{answer}
			\begin{proof}
			Assume that $T$ is a linear transformation. Then by definition, $T(\vec{0} + \vec{0}) = T(\vec{0}) + T(\vec{0})$. However, $\vec{0} + \vec{0} = \vec{0}$. Therefore, $T(\vec{0} + \vec{0}) = T(\vec{0})$. Thus, we have that $T(\vec{0}) = T(\vec{0}) + T(\vec{0})$. Therefore, $T(\vec{0}) = \vec{0}$. 
			\end{proof}
		\end{answer}
		\item Consider the matrices $A = \mat{1 & 0 \\ k & 1}$, and $B = \mat{2 & 0 \\ 0 & 1}$.
		
		\begin{enumerate}
			\item Describe the geometric effect of $A$.  Feel free to use the language of the in-class worksheets.
			\begin{answer}
				Let
				\[
				\mat{x\\y} \in \R^2.
				\]
				Then, 
				\[
				A\mat{x\\y} = \mat{1&0\\k&1}\mat{x\\y} = \mat{x\\kx+y}.
				\]
				By looking at the action of $A$ on the vector $[x,y]$, we see $A$ is a "sheer transformation", which keeps the first component of the vector the same and adds a $k$-multiple of the first component to the second component. 
			\end{answer}
			\item Compute the product $BA$.
			\begin{answer}
				\[
				BA = \mat{2&0\\0&1}\mat{1&0\\k&1} = \mat{2&0\\k&1}.
				\]
			\end{answer}
			\item Find a matrix $C$ such that $AC = BA$.
			\begin{answer}
				Let $C$ be the matrix
				\[
				\mat{a & b \\ c & d}
				\]
				with $a,b,c,d \in \R$. Then, we want
				\[
				\mat{1&0\\k&1}\mat{a&b\\c&d} = \mat{2 & 0 \\ k&1}.
				\]
				This gives a system of equations. 
				\[
				\begin{cases}
					a &= 2 \\
					b &= 0 \\
					ka + c &= k \\
					kb + d &= 1
				\end{cases} \iff
				\begin{cases}
					a &= 2 \\
					b &= 0 \\
					c &= -k \\
					d &= 1
				\end{cases} \implies C = \mat{2 & 0 \\ -k & 1}.
				\]
			\end{answer}
			\item Describe geometrically why $C \neq B$.
			\begin{answer}
				Let $[x,y] \in \R^2$. Then,
				\[
				C\mat{x\\y} = \mat{2 & 0 \\ -k & 1}\mat{x\\y} = \mat{2x\\-kx + y},
				\]
				but
				\[
				B\mat{x\\y} = \mat{2&0\\0&1}\mat{x\\y} = \mat{2x\\y}.
				\]
				Both transformations $B$ and $C$ double the first component of the vector $\vec{x} = [x,y]$, but while $B$ doesn't change the second component, $C$ subtracts $k$ multiples of the first component from the second component. Hence, unless $k = 0$, $B$ and $C$ must be different. 
			\end{answer}
		\end{enumerate}
		
		\item Suppose $A$ is an unknown matrix, but $A^{-1} = \mat{3 & 2 & -1 \\ -2 & 0 & 1 \\ 2 & 1 & 1}$
		
		\begin{enumerate}
			\item Find a solution to the matrix equation $A \vec{x} = \mat{2 \\ 1 \\ 6}$.
			\begin{answer}
			\[
			A\vec{x} = \mat{2\\1\\6} \iff A^{-1} A \vec{x} = A^{-1} \mat{2\\1\\6} \iff I\vec{x} = A^{-1}\mat{2\\1\\6}. 
			\]
			Thus,
			\[
			\vec{x} = \mat{3 & 2 & -1 \\ -2 & 0 & 1 \\ 2 & 1 & 1}\mat{2\\1\\6} = \mat{2\\2\\11} \implies \vec{x} = \mat{2\\2\\11}.
			\]
			\end{answer}
			\item Are there other solutions besides the one you found in (a)?  Explain how you know.
			\begin{answer}
				There are no other solutions. The matrix $A$ has an inverse, and is thus invertible. Therefore, by the Invertible Matrix Theorem, the equation
				\[
				A\vec{x} = \mat{2\\1\\6}
				\]
				has a unique solution. 
			\end{answer}
		\end{enumerate}
		
		
		
		
	\end{enumerate}

\end{document}
