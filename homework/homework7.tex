\documentclass[letter]{article}
\usepackage{amsmath}
\usepackage{amsfonts}
\usepackage{amssymb}
\usepackage{ifthen}
\usepackage{fancyhdr}
\usepackage{enumitem}

%%%
% Set up the margins to use a fairly large area of the page
%%%
\oddsidemargin=.2in
\evensidemargin=.2in
\textwidth=6in
\topmargin=0in
\textheight=9.0in
\parskip=.07in
\parindent=0in
\pagestyle{fancy}

%%%
% Set up the header
%%%
\newcommand{\setheader}[6]{
	\lhead{{\sc #1}\\{\sc #2} ({\small \it \today})}
	\rhead{
		{\bf #3} 
		\ifthenelse{\equal{#4}{}}{}{(#4)}\\
		{\bf #5} 
		\ifthenelse{\equal{#6}{}}{}{(#6)}%
	}
}

%%%
% Set up some shortcut commands
%%%
\newcommand{\R}{\mathbb{R}}
\newcommand{\N}{\mathbb{N}}
\newcommand{\Z}{\mathbb{Z}}
\newcommand{\Proj}{\mathrm{proj}}
\newcommand{\Perp}{\mathrm{perp}}
\newcommand{\proj}{\mathrm{proj}}
\newcommand{\Span}{\mathrm{span}}
\newcommand{\Null}{\mathrm{null}}
\newcommand{\Rank}{\mathrm{rank}}
\newcommand{\Det}{\mathrm{det}}
\newcommand{\mat}[1]{\begin{bmatrix}#1\end{bmatrix}}
\renewcommand{\d}{\mathrm{d}}

%%%
% This is where the body of the document goes
%%%
\begin{document}
	\setheader{Math 240}{Homework 7}{Not for turning in}{}{}{}

	These problems will not be turned in nor graded.  However, a non-empty subset of these
	problems will appear on the final skills check/takehome.  The numbers may change when
	they appear on the exam, but it behooves you to have a thorough understanding of every problem on this homework.
	
	\begin{enumerate}
		
		\item Suppose the matrix equation $A\vec x=\mat{3\\2\\7}$ has the general solution
			\[
				\vec x=\mat{1\\0\\0}+s\mat{1\\1\\0}+t\mat{-1\\0\\1}.
			\]
			\begin{enumerate}
				\item How many rows and how many columns does $A$ have?
				\item Find $\Null(A)$.
				\item Find $\Rank(A)$.
				\item Find $\text{col}(A)$.
				\item Find $\text{row}(A)$.
			\end{enumerate}
		
		\item Let
			\[
				\vec b_1=\mat{1\\1\\1}\qquad\vec b_2=\mat{1\\-1\\0}\qquad
				\vec b_3=\mat{-1\\0\\1}\qquad \vec c=\mat{1\\2\\3}
			\]
			and let $\mathcal B=\{\vec b_1,\vec b_2,\vec b_3\}$ and $\mathcal S=\{\vec e_1,\vec e_2,\vec e_3\}$.
			Suppose $T:\R^3\to\R^3$ is a linear transformation and $T(\vec b_1)=2\vec b_1$, 
			$T(\vec b_2)=3\vec b_2$, and $T(\vec b_3)=-\vec b_3$.
			\begin{enumerate}
				\item Compute $[\vec c]_{\mathcal B}$.
				\item Compute $[T\vec c]_{\mathcal B}$ and $[T\vec c]_{\mathcal S}$.
				\item Find a matrix for $T$ in the $\mathcal B$ basis  and a matrix for $T$ in
					the $\mathcal S$ basis. (In class, we might have said, ``A matrix for
					$T$ in the $\mathcal B$ coordinate system.''  This is another way of 
					saying, ``A matrix for $T$ in the $\mathcal B$ basis.'')
			\end{enumerate}
		
		\item Read section 3.1 and 3.2 in your textbook about computing the determinant of a matrix.
			Let $A=\mat{1&2&1\\1&1&1\\1&0&0}$ and $B=\mat{1&2&1\\1&1&1\\1&0&x}$.
		\begin{enumerate}
			\item Compute $\Det(A)$.
			\item Compute $\Det(B)$.  For what values of $x$ is $B$ not invertible?
		\end{enumerate}

		\item Let $A=\mat{1&2\\5&9}$.
		\begin{enumerate}
			\item Find an equation for the function $p(x)=\Det(A-xI)$ (this is called the
				\emph{characteristic polynomial} of $A$).
			\item For what values of $x$ is $A-xI$ non-invertible?
			\item Compute $p(A)$, the polynomial $p$ with the matrix $A$ plugged into it.  When you plug a matrix
				into a polynomial, replace any constant terms $k$ with the matrix $kI$.
				Can you guess
				why $p$ is called an \emph{annihilating} polynomial for $A$?
		\end{enumerate}

		\item For each of the following, either give an example or a reason why it is impossible.
		\begin{enumerate}
			\item A linear transformation $T:\R^2\to\R^2$ that is invertible.
			\item A linear transformation $T:\R^2\to\R^2$ that is not invertible.
			\item A non-linear transformation $T:\R^2\to\R^2$.
			\item A linear transformation $T:\R^2\to\R^2$ whose null space equals its range.
			\item A linear transformation $T:\R^2\to\R^2$ that when represented as a matrix in
			the standard basis has a column space equal to its row space.
			\item A linear transformation $T:\R^2\to\R^2$ so that $T^2$ is the identity, but $T$ is not invertible.
			\item A linear transformation $T:\R^2\to\R^2$ so that $T^3$ is the identity, but $T^2$ is not the identity.
			\item A linear transformation $T:\R^2\to\R^2$ with exactly \emph{one} eigenvector.
			\item A linear transformation $T:\R^2\to\R^2$ with exactly \emph{one} eigen direction (i.e., all eigenvectors lie on a single line).
			\item A linear transformation $T:\R^2\to\R^2$ with exactly \emph{two} eigen directions (i.e., all eigenvectors lie on one of two lines).
			\item A linear transformation $T:\R^2\to\R^2$ with \emph{infinitely many} eigen directions.
			\item A linear transformation $T:\R^2\to\R^2$ with no real eigenvectors.
			\item A linear transformation $T:\R^2\to\R^2$ with eigenvalues $3$ and $-2$.
			\item A linear transformation $T:\R^2\to\R^2$ where $\mat{1\\1}$ is an eigenvector with eigenvalue $2$ and $\mat{2\\3}$
				is an eigenvector with eigenvalue $0$.
		\end{enumerate}

	\item {\it This problem will not be on the exam but is included for you to stress-test your
		mathematical thinking.}
		
		Let $T:\R^n\to\R^n$. A subspace $X\subseteq \R^n$ is called \emph{invariant with respect to $T$} if $T(X)=X$.  That is,
		       $\{\vec v:\vec v=T\vec x\text{ for some }\vec x\in X\}=X$.  Note, this is \emph{different} than saying $T\vec x=\vec x$ for all $\vec x\in X$.
		\begin{enumerate}
		       \item Describe all invariant subspaces of the linear transformation given by the matrix $\mat{1&0\\0&2}$.
		       \item Describe all invariant subspaces of the linear transformation given by the matrix $\mat{2&0\\0&2}$.
		       \item Describe all invariant subspaces of the sheer given by the matrix $\mat{1&a\\0&1}$ where $a\neq 0$.
			\item Your friend from another university proposes the following addition to the invertible matrix theorem:
				\begin{quote}
					An $n\times n$ matrix $A$ is invertible if and only if $\R^n$ is an invariant subspace of the transformation
					given by $A$.
				\end{quote}
				Is he right?  If so, prove it is correct.  If not, give a counterexample.
			\item Suppose $T:\R^2\to\R^2$ has $\{\vec 0\}$, $\left\{\mat{x\\y}:y=2x\right\}$, and $\R^2$ as its \emph{only}
				invariant subspaces.  Give an example of a vector $\vec v$ that is an eigenvector for $T$ and a vector $\vec w$
				that is \emph{not} an eigenvector for $T$.  Explain how you know.
		\end{enumerate}
		
		
	\end{enumerate}

\end{document}
