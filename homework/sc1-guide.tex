\documentclass[10pt]{article}
\usepackage{amsmath, amsfonts, amsthm}
\usepackage{fancyhdr,parskip}
\usepackage{fullpage}
\usepackage[answerdelayed]{exercise}


\setlength{\headheight}{15.2pt}
\setlength{\headsep}{20pt}
\pagestyle{fancyplain}

\renewcommand{\ExerciseHeader}{\textbf{Problem \ExerciseHeaderNB .}}
\renewcommand{\AnswerHeader}{\textbf{Problem \ExerciseHeaderNB .}}

\newcommand{\be}[1]{\begin{Exercise}[label=#1]}
\newcommand{\ee}{\end{Exercise}}
\newcommand{\ba}[1]{\begin{Answer}[ref=#1]}
\newcommand{\ea}{\end{Answer}}
\newcommand{\ds}{\displaystyle}
\newcommand{\is}{\hspace{2pt}}
\newcommand{\dx}{\is dx}

\newcommand{\Z}{\mathbb{Z}}
\newcommand{\Q}{\mathbb{Q}}
\newcommand{\R}{\mathbb{R}}
\newcommand{\C}{\mathbb{C}}
\newcommand{\F}{\mathbb{F}}


\fancyhead[L]{Skills Check 1 Guide}
\fancyhead[C]{}
\fancyhead[R]{Math 240, Fall 2016}


\begin{document}

For Skills Check 1, you should be able to give a mathematically precise definition of the following:
\begin{itemize}
	\item linear combination
	\item linearly independent/dependent
	\item span
	\item consistent/inconsistent
	\item free variable
	\item trivial/nontrivial linear combination
	\item \ldots and any other terminology or notation we've discussed in class.
\end{itemize}

You should be able to do the following computations:
\begin{itemize}
	\item Do vector arithmetic: addition, subtraction, scalar multiplication, and dot product.
	\item Multiply matrices.
	\item Express a vector as a linear combination of other vectors.
	\item Determine whether or not a set of vectors is linearly independent.
	\item Determine whether one vector is in the span of others.
	\item Find the span of a set of vectors.
	\item Put a matrix into reduced row echelon form, and recognize whether or not it is.
	\item Convert a system of linear equations into augmented-matrix form and matrix-equation (Av=b) form, and vice versa.
	\item Find all solutions to a system of linear equations.
\end{itemize}





\end{document}
