\documentclass[letter]{article}
\usepackage{amsmath}
\usepackage{amsfonts}
\usepackage{amssymb}
\usepackage{ifthen}
\usepackage{fancyhdr}
\usepackage{enumitem}

%%%
% Set up the margins to use a fairly large area of the page
%%%
\oddsidemargin=.2in
\evensidemargin=.2in
\textwidth=6in
\topmargin=0in
\textheight=9.0in
\parskip=.07in
\parindent=0in
\pagestyle{fancy}

%%%
% Set up the header
%%%
\newcommand{\setheader}[6]{
	\lhead{{\sc #1}\\{\sc #2} ({\small \it \today})}
	\rhead{
		{\bf #3} 
		\ifthenelse{\equal{#4}{}}{}{(#4)}\\
		{\bf #5} 
		\ifthenelse{\equal{#6}{}}{}{(#6)}%
	}
}

%%%
% Set up some shortcut commands
%%%
\newcommand{\R}{\mathbb{R}}
\newcommand{\N}{\mathbb{N}}
\newcommand{\Z}{\mathbb{Z}}
\newcommand{\Proj}{\mathrm{proj}}
\newcommand{\Perp}{\mathrm{perp}}
\newcommand{\proj}{\mathrm{proj}}
\newcommand{\Span}{\mathrm{span}}
\newcommand{\Null}{\mathrm{null}}
\newcommand{\Rank}{\mathrm{rank}}
\newcommand{\mat}[1]{\begin{bmatrix}#1\end{bmatrix}}
\renewcommand{\d}{\mathrm{d}}

%%%
% This is where the body of the document goes
%%%
\begin{document}
	\setheader{Math 240}{Homework 5}{Due: Thursday, November 3}{}{}{}
	\emph{We are nearly experts at typing by now.  Though you are not required
	to type your homework, it's strongly encouraged.  You can even download
	the \texttt{tex} file for this homework and type your answers below
	each problem.  Using the} \verb!\begin{quote}! \emph{and} \verb!\end{quote}! \emph{environment
		will indent anything you type inbetween.  Perfect for typing answers!}
	\begin{enumerate}
		\item Let $A$ and $B$ be $n\times n$ invertible matrices and let $X=AB$.  Does
			$X^{-1}=A^{-1}B^{-1}$ or does $X^{-1}=B^{-1}A^{-1}$ or neither? Explain.
		\item For each of the following sets, determine whether or
			not it is a subspace.  Explain your answer.
			
		\begin{enumerate}
			\item ${A}=\left\{\vec x:\vec x\cdot \mat{2\\3}=0\right\}\subseteq \R^2$ 
			\item ${B}\subseteq \R^3$ is the $x$-axis.
			\item ${C}\subseteq \R^3$ is the plane given in vector form as $\vec x=t\mat{1\\1\\1}+s\mat{-2\\3\\-1}+\mat{0\\1\\5}$.
			\item ${D}\subseteq \R^3$ is the plane with normal vector $\vec n=\mat{1\\1\\-1}$ passing through
				the point $\mat{2\\2\\4}$.
			\item ${E}=\{(x,y):y=3x+4\}\subseteq \R^2$.
			\item ${F}=\Span\{\vec u_1, \vec u_2\}\subseteq \R^2$ where $\vec u_1=\mat{1\\0}$ and $\vec u_2=\mat{2\\0}$.
			\item ${G}\subseteq \R^4$ is the set of all solutions to the matrix equation $\mat{1&1&0&-2\\0&3&0&0\\0&1&0&0}\vec x=\mat{0\\0\\0}$.
			\item ${H}=\{(x,y): xy=0\}\subseteq \R^2$.
		\end{enumerate}

		\item For every set in problem 1 that is a vector space, find a basis.
		\item For every set in problem 1 that is a vector space, find its dimension.

		\item Let $\mathcal P$ be the plane in $\R^3$ given in vector form by
			$\vec x=t\vec d_1+s\vec d_2+\vec p$ for unknown vectors $\vec d_1,\vec d_2,\vec p$.
			\begin{enumerate}
				\item Show that $\mathcal P$ is a subspace if and only if $\vec p=\vec 0$.  That is, show
				that if $\vec p=\vec 0$, then $\mathcal P$ is a subspace, and if $\vec p\neq \vec 0$,
				then $\mathcal P$ is not a subspace.
				\item Suppose that $\vec d_1$ and $\vec d_2$ were accidentally chosen
					to be linearly dependent.  Is it still the case that $\mathcal P$ is
					a subspace exactly when $\vec p=\vec 0$, or is the outcome different now?
			\end{enumerate}
		\item Let $\mathcal V$ be the subspace spanned by $\vec v_1=\mat{1\\2\\1},\vec v_2=\mat{1\\1\\0},\vec v_3=\mat{4\\0\\-4},
			\vec v_4=\mat{2\\3\\1},\vec v_5=\mat{-5\\4\\9}$.
			\begin{enumerate}
				\item Find a basis for $\mathcal V$ and call your basis vectors $\vec b_1$, $\vec b_2$, etc.
				\item Describe $\mathcal V$ geometrically.
				\item Let $V=[\vec v_1|\vec v_2|\vec v_3|\vec v_4|\vec v_5]$ be the matrix whose columns are the vectors
					$\vec v_1,\vec v_2,\vec v_3,\vec v_4,\vec v_5$, let $B=[\vec b_1|\vec b_2|\cdots]$ be the matrix
					whose columns are your basis vectors from part (a), and let $\vec v\in \mathcal V$.

					Without computing, how many solutions does the equation $V\vec x=\vec v$ have?  How
					about $B\vec x=\vec v$?
			\end{enumerate}
		\item Suppose $A$ is an invertible matrix and $\vec c_1,\vec c_2,\ldots,\vec c_n$ are its columns.  Is
			$\{\vec c_1,\vec c_2,\ldots, \vec c_n\}$ a basis?  Describe $\Span\{\vec c_1,\vec c_2,\ldots, \vec c_n\}$.  Explain
			your reasoning.

	\end{enumerate}

\end{document}
