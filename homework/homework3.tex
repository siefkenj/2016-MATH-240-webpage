\documentclass[letter]{article}
\usepackage{amsmath}
\usepackage{amsfonts}
\usepackage{amssymb}
\usepackage{ifthen}
\usepackage{fancyhdr}
\usepackage{enumitem}

%%%
% Set up the margins to use a fairly large area of the page
%%%
\oddsidemargin=.2in
\evensidemargin=.2in
\textwidth=6in
\topmargin=0in
\textheight=9.0in
\parskip=.07in
\parindent=0in
\pagestyle{fancy}

%%%
% Set up the header
%%%
\newcommand{\setheader}[6]{
	\lhead{{\sc #1}\\{\sc #2} ({\small \it \today})}
	\rhead{
		{\bf #3} 
		\ifthenelse{\equal{#4}{}}{}{(#4)}\\
		{\bf #5} 
		\ifthenelse{\equal{#6}{}}{}{(#6)}%
	}
}

%%%
% Set up some shortcut commands
%%%
\newcommand{\R}{\mathbb{R}}
\newcommand{\N}{\mathbb{N}}
\newcommand{\Z}{\mathbb{Z}}
\newcommand{\Proj}{\mathrm{proj}}
\newcommand{\Perp}{\mathrm{perp}}
\newcommand{\proj}{\mathrm{proj}}
\newcommand{\Span}{\mathrm{span}}
\newcommand{\Null}{\mathrm{null}}
\newcommand{\Rank}{\mathrm{rank}}
\newcommand{\mat}[1]{\begin{bmatrix}#1\end{bmatrix}}

%%%
% This is where the body of the document goes
%%%
\begin{document}
	\setheader{Math 281-1}{Homework 3}{Due: Thursday, October 20}{}{}{}
	\begin{enumerate}
		\item Let $\vec a=\mat{1\\2\\3}$, $\vec b=\mat{4\\5\\6}$, $\vec c=\mat{d\\1\\1}$, and $\vec w=\mat{w_1\\w_2\\w_3}$
			be vectors where $d,w_1,w_2,w_3\in\R$ are unknown constants.
			\begin{enumerate}
			\item For what values of $d$ is $\{\vec a,\vec b,\vec c\}$ linearly
			independent?  For which values of $d$ is $\{\vec a,\vec b,\vec c\}$ linearly dependent?
			\item Write down the system of equations coming from the rows of the vector equation
				\[
					x\vec a+y\vec b+z\vec c=\vec w.
				\]
			\item Give three numeric examples of different vectors $\vec w$ such that the above system is consistent
				no matter what $d$ is. Explain.
			\item Give a numeric example of a vector $\vec w$ such that the above system is only 
				consistent for some $d$. Explain.
			\end{enumerate}
		\item \begin{enumerate}
				\item Use an augmented matrix to solve
					\begin{align*}
						x+y&=7\\
						2x-3y&=13.
					\end{align*}
					Are there any values you could replace the right hand side of the equations with
					such that there would be no solution?  Explain both \emph{geometrically}
					(using vectors, span, etc.) and \emph{algebraically} (using systems, consistency,
					etc.) using technical linear algebra terms.
				\item 
				Consider the system given by the augmented matrix
				\[
					C=\left[\begin{array}{ccccc|c}
						1&0&1&2&0&-1\\
						0&1&1&0&0&3\\
						0&0&0&0&1&4
					\end{array}\right].
				\]
				and call the variables in this system $x_1,x_2,
				x_3,x_4,x_5$.  Write all solutions to this system in vector form.
				How many free variables are there?
				\item There are 10 ways to pick two things from the set $\{x_1,x_2,x_3,x_4,x_5\}$.
					For each of the ten ways, determine whether that pair is a valid choice
					of free variables for $C$.
				\item Write down all solutions to the homogeneous system corresponding to $C$ 
					(i.e., when the right-hand side is replaced with all zeros).  How does this
					set of solutions compare to the set of solutions of $C$?
			\end{enumerate}
		\item Let $M=\mat{3&2&3\\0&0&1\\2&1&0}$ and $\vec w=\mat{1\\2\\-1}$.
			\begin{enumerate}
				\item Find solutions $\vec v_1$, $\vec v_2$, and $\vec v_3$ to the three matrix
					equations
					\[
						M\vec v_1=\vec e_1\qquad M\vec v_2=\vec e_2\qquad M\vec v_3=\vec e_3.
					\]
				\item Compute $M(a\vec v_1)$ and $M(a\vec v_1+b\vec v_2)$ (where $\vec v_i$ are from above)
					where $a,b\in\R$ are unknown scalars.
					Was what happened a surprise?
				\item Express the solution to the matrix equation $M\vec x=\vec w$ as a linear combination
					of $\vec v_1$, $\vec v_2$, and $\vec v_3$.  (Recall, $\vec w$ is defined at the beginning
					of the problem.)
				\item Let $V=[\vec v_1|\vec v_2|\vec v_3]$ be the matrix whose columns are $\vec v_1$, $\vec v_2$,
					and $\vec v_3$.  Compute the matrix product $MV$.  Explain why you got the
					result you did.
				\item Can you use $V$ to help you solve the system
					\[
						M\vec x=\mat{a\\b\\c}?
					\]
					If so, explain how and do so.
				\item Compute the matrix product $VM$.  Can you explain why you got what you did?
					(\emph{Hint: you might have to think about linear transformations for this one.})
			\end{enumerate}

		\item Consider the transformations $A:\R^2\to \R^2$ and $B:\R^2\to \R^2$ given by
			the formulas
			\[
				A(x,y) = (x-y, x+y)\qquad\text{and}\qquad B(x,y)=(x^2,y^2).
			\]
			\begin{enumerate}
				\item Compute $A(\vec e_1)$, $A(\vec e_2)$, $A(\vec e_1+\vec e_2)$,
					$B(\vec e_1)$, $B(\vec e_2)$, and $B(\vec e_1+\vec e_2)$.
				\item Find a matrix $M_A$ so that $A$ is given by matrix
					multiplication, or explain why it is impossible.
				\item Find a matrix $M_B$ so that $B$ is given by matrix
					multiplication, or explain why it is impossible.
				\item A function $X:\R^n\to\R^m$ is called \emph{linear} if it satisfies:
					\begin{enumerate}
						\item $X(\alpha \vec v) = \alpha X(\vec v)$ for all $\vec v\in \R^n$
					and all $\alpha\in \R$ {\bf and} 
						\item $X(\vec v+\vec w) = X(\vec v)
					+X(\vec w)$ for all $\vec v, \vec w\in \R^n$.
					\end{enumerate}

					For each of $A$ and $B$, determine whether or not it is a linear function. Prove your answers.

			\end{enumerate}
		\item \begin{enumerate}
				\item Let $\vec v\in \R^n$ and define the function $d_{\vec v}:\R^n\to\R$ by
					$d_{\vec v}(\vec w) = \vec v\cdot \vec w$.  Prove that $d_{\vec v}$
					is linear.  (\emph{Hint: you should be having flashbacks to homework 1}.)
				\item For a $2\times 2$ matrix $M$, let $f_{M}:\R^2\to\R^2$ be defined by
					$f_M(\vec v) = M\vec v$, where $\vec v$ is a column vector.  Prove that
					$f_M$ is a linear transformation.
				\item Make a conjecture about functions that can be computed using matrix
					multiplication and their linearity.
			\end{enumerate}
	\end{enumerate}

\end{document}
