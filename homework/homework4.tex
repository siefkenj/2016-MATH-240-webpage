\documentclass[letter]{article}
\usepackage{amsmath}
\usepackage{amsfonts}
\usepackage{amssymb}
\usepackage{ifthen}
\usepackage{fancyhdr}
\usepackage{enumitem}

%%%
% Set up the margins to use a fairly large area of the page
%%%
\oddsidemargin=.2in
\evensidemargin=.2in
\textwidth=6in
\topmargin=0in
\textheight=9.0in
\parskip=.07in
\parindent=0in
\pagestyle{fancy}

%%%
% Set up the header
%%%
\newcommand{\setheader}[6]{
	\lhead{{\sc #1}\\{\sc #2} ({\small \it \today})}
	\rhead{
		{\bf #3} 
		\ifthenelse{\equal{#4}{}}{}{(#4)}\\
		{\bf #5} 
		\ifthenelse{\equal{#6}{}}{}{(#6)}%
	}
}

%%%
% Set up some shortcut commands
%%%
\newcommand{\R}{\mathbb{R}}
\newcommand{\N}{\mathbb{N}}
\newcommand{\Z}{\mathbb{Z}}
\newcommand{\Proj}{\mathrm{proj}}
\newcommand{\Perp}{\mathrm{perp}}
\newcommand{\proj}{\mathrm{proj}}
\newcommand{\Span}{\mathrm{span}}
\newcommand{\Null}{\mathrm{null}}
\newcommand{\Rank}{\mathrm{rank}}
\newcommand{\mat}[1]{\begin{bmatrix}#1\end{bmatrix}}
\renewcommand{\d}{\mathrm{d}}

%%%
% This is where the body of the document goes
%%%
\begin{document}
	\setheader{Math 281-1}{Homework 4}{Due: Thursday, November 5}{}{}{}
	\begin{enumerate}
		\item 
		\begin{enumerate}
			\item A valley can modeled by the equation $z=f(x,y)=3e^{xy}$ where $x$ is kilometers
			north and $y$ is kilometers east.  From a helicopter you observe a
			hiking trail that appears to be modeled by $x^3+y^3=27$.  Find the highest and
			lowest points along the hiking trail.

		\item Write down an equation of the tangent plane (in vector form) at the minimum and maximum points
			along the hiking trail.
		\end{enumerate}
		
		\item Let $\displaystyle A=\int_{x=0}^{x=6}\int_{y=x/3}^{y=2} x\sqrt{y^3+1}\,\d y \d x$.
		\begin{enumerate}
			\item Find a region $R$ and a function $f$ so that $\displaystyle A=\iint_R f(x,y) \,\d V$.
			\item Find $A$.
		\end{enumerate}
		
		\item Let $C$ be a circular path in $\R^3$ that passes through the points
			$\vec a = (1,2,3)$, $\vec b = (1,1,3)$, and $\vec c = (-1,1,1)$.
			Let $f:\R^3\to\R$ be distance from the origin.  That is, $f(\vec x) = \|x\|$.
		\begin{enumerate}
			\item Find the radius of $C$.  Then find two functions $g:\R^3\to \R$ and
				$h:\R^3\to \R$ so that $g(\vec x_0) = h(\vec x_0) = 0$ if and only if
				$\vec x_0\in C$.  (Hint, think about how to make two surfaces intersect
				at exactly $C$.)
			\item Could you find a single function $r:\R^3\to \R$ 
				so that $r(\vec x_0)=0$ if and only if $\vec x_0\in C$?  Why or why not?
			\item Find a tangent vector to $C$ at the point $\vec c$, and call this 
				vector $T_{\vec c}$.  Write down vector form of the plane of vectors
				orthogonal to $T_{\vec c}$.
			\item Moving along $C$ from the point $\vec c$, how fast is $f$ changing
				\emph{with respect to distance}?
			\item Find the minimum and maximum values $f$ attains along the curve $C$.
				(Hint, it's much easier to find the minimum and maximum values
				of $f^2$ and use those to minimize/maximize $f$.)
		\end{enumerate}


		\item Consider the stretched polar coordinates system $\mathcal {SP}$.
			Coordinates in $\mathcal {SP}$ are given by pairs $(\rho, \phi)$ and relate to
			rectangular coordinates via the equations $x=\rho \cos \phi$ and $y=2\rho\sin \phi$ where $\rho\in[0,\infty)$
				and $\phi\in[0,2\pi)$.
		\begin{enumerate}
			\item Draw the level curves $\rho = m$ for $m\in \{0,1,2,3\}$ and $\phi=n$ 
				for $n\in\{0,\pi/8,\pi/4,3\pi/8,\pi/2\}$.
			\item Compute the area enclosed by the curves $\rho=\rho_0$, $\rho=\rho_0+\Delta \rho$,
				$\phi=\phi_0$, and $\phi=\phi_0+\Delta\phi$ and call this area $\Delta V(\rho_0,\phi_0)$.
			\item Give an isometric parameterization for{\ldots}just kidding. Instead of forcing a parameterization
				to preserve area, let's measure how much it changes area.  We'd like to compute
				the \emph{density} of the coordinate system $\mathcal{SP}$.  Consider the following: 
				Let $G_n=\{(\frac{i}{n},\frac{j}{n}):i,j\in \Z\}$ be a grid of regularly spaced points.
				If we interpret $G_n$ as points given in rectangular coordinates and we look at a
				square of size $\frac{1}{n}\times \frac{1}{n}$, we expect to find one point in that square.
				That is, if $S_n(x,y)=[x,x+\frac{1}{n})\times[y,y+\frac{1}{n})$ is the square of width $\frac{1}{n}$
				with lower left corner at the point $(x,y)$, then the size of $S_n(x,y)\cap G_n=1$ regardless of $(x,y)$ and
				regardless of $n$.  However, if we interpret $G_n$ as specified in $\mathcal{SP}$ coordinates,
				suddenly the size of $S_n(x,y)\cap G_n$ depends on $(x,y)$ and on $n$. The \emph{density} of $\mathcal{SP}$
				coordinates at the point $(x,y)$ is defined as
				\vspace{-.2cm}
				\[
					\mathrm{den}(x,y) = \lim_{n\to\infty}\text{expected size of }S_n(x,y)\cap G_n,
				\]
				when $G_n$ is interpreted in $\mathcal {SP}$ coordinates.
				Alternatively (if the \emph{expected size} language is too ambiguous), let
				$f:\R^2\to\R^2$ be the function defined by $(a,b)\mapsto (a\cos b, 2a\sin b)$
				that takes $\mathcal{SP}$ coordinates to rectangular coordinates.
				Then,
				\[
					\mathrm{den}(x,y) = \lim_{n\to\infty}\frac{\text{area of }S_n(x,y)}{\text{area of }f(S_n(x,y))},
				\]
				where $f(S_n(x,y))$ is the image of $S_n(x,y)$ under $f$.

				\vspace{.3cm}
				Compute the density of $\mathcal{SP}$ coordinates.
		\end{enumerate}

	\end{enumerate}

\end{document}
