\documentclass[letter]{article}
\usepackage{amsmath}
\usepackage{amsfonts}
\usepackage{amssymb}
\usepackage{ifthen}
\usepackage{fancyhdr}
\usepackage{enumitem}

%%%
% Set up the margins to use a fairly large area of the page
%%%
\oddsidemargin=.2in
\evensidemargin=.2in
\textwidth=6in
\topmargin=0in
\textheight=9.0in
\parskip=.07in
\parindent=0in
\pagestyle{fancy}

%%%
% Set up the header
%%%
\newcommand{\setheader}[6]{
	\lhead{{\sc #1}\\{\sc #2} ({\small \it \today})}
	\rhead{
		{\bf #3} 
		\ifthenelse{\equal{#4}{}}{}{(#4)}\\
		{\bf #5} 
		\ifthenelse{\equal{#6}{}}{}{(#6)}%
	}
}

%%%
% Set up some shortcut commands
%%%
\newcommand{\R}{\mathbb{R}}
\newcommand{\N}{\mathbb{N}}
\newcommand{\Z}{\mathbb{Z}}
\newcommand{\Proj}{\mathrm{proj}}
\newcommand{\Perp}{\mathrm{perp}}
\newcommand{\proj}{\mathrm{proj}}
\newcommand{\Span}{\mathrm{span}}
\newcommand{\Null}{\mathrm{null}}
\newcommand{\Rank}{\mathrm{rank}}
\newcommand{\mat}[1]{\begin{bmatrix}#1\end{bmatrix}}

%%%
% This is where the body of the document goes
%%%
\begin{document}
	\setheader{Math 240}{Homework 4}{Due: Thursday, October 27}{}{}{}
	\begin{enumerate}
		
		\item \textbf{Reading Assignment:} Read and practice the algorithm for finding the inverse of a matrix, found at the end of section 2.2 of the textbook.
		
		\item For each of the following transformations, find a matrix corresponding to the transformation.
		
		\begin{enumerate}
			\item $\mathcal{A}$: Rotate $90^\circ$ counterclockwise around the origin (in $\R^2$).
			\item $\mathcal{B}$: Send every vector in $\R^2$ to the zero vector.
			\item $\mathcal{C}$: Project (in $\R^3$) onto the $yz$-plane.
			\item $\mathcal{D}$: Project (in $\R^3$) onto the $x$-axis.
			\item $\mathcal{E}$: Reflect (in $\R^3$) across the $xy$-plane.
			\item $\mathcal{F}$: Stretch by a factor of 2 in the $y$-direction (in $\R^2$).
			\item $\mathcal{G}$: Stretch by a factor of 2 in the $y$-direction (in $\R^3$).
			\item $\mathcal{H}$: Rotate $90^\circ$ counterclockwise (as viewed looking ``down'' from the positive $y$-axis towards the origin) around the $y$-axis (in $\R^3$).
		\end{enumerate}
		
		
		\item \begin{enumerate}
			\item For each transformation in problem 2, explain geometrically whether or not the transformation is invertible.
			\item Which of the matrices you generated in problem 2 are invertible?  (You may wish to use the algorithm you read about.)
			\item Suppose $f$ is a transformation that is represented by the matrix $M$.
				Is it possible for $f$ to be invertible and $M$ not invertible?  Is it possible
				for $M$ to be invertible and $f$ not be invertible? Explain your reasoning.
		\end{enumerate}
		
		
		\item Each of the following matrices gives a transformation of $\R^2$ or $\R^3$.  
			Describe the effect of that transformation in terms of the transformations from problem 2.
			(Hint: you may need to combine two, three, or more transformations from problem 2 for some
			examples.)
			\[
				X=\mat{0 & -1 \\ 2 & 0}
				\qquad \qquad
				Y=\mat{0 & 0 & 0\\ 0 & 1 & 0 \\ 0 & 0 & -1}
			\]
			\[
				Z=\mat{0 & 0 & 1 \\ 0 & 2 & 0 \\ -1 & 0 & 0}
				\qquad\qquad
				W=\mat{0 & 0 & 1 \\ 0 & 0 & 0 \\ 0 & 0 & 0}
			\]

		
		\item Use the definition of a linear transformation
			to show that if $T$ is a linear transformation, then $T(\vec{0}) = \vec{0}$.  
			Do not appeal to the matrix representation of $T$.
	
		
		\item Consider the matrices $A = \mat{1 & 0 \\ k & 1}$, and $B = \mat{2 & 0 \\ 0 & 1}$.
		
		\begin{enumerate}
			\item Describe the geometric effect of $A$.  Feel free to use the language of the in-class worksheets.
			\item Compute the product $BA$.
			\item Find a matrix $C$ such that $AC = BA$.
			\item Describe geometrically why $C \neq B$.
		\end{enumerate}
		
		\item Suppose $A$ is an unknown matrix, but $A^{-1} = \mat{3 & 2 & -1 \\ -2 & 0 & 1 \\ 2 & 1 & 1}$
		
		\begin{enumerate}
			\item Find a solution to the matrix equation $A \vec{x} = \mat{2 \\ 1 \\ 6}$.
			\item Are there other solutions besides the one you found in (a)?  Explain how you know.
		\end{enumerate}
		
		
		
		
	\end{enumerate}

\end{document}
