\documentclass[10pt]{article}
\usepackage{amsmath, amsfonts, amsthm}
\usepackage{fancyhdr,parskip}
\usepackage{fullpage}
\usepackage[answerdelayed]{exercise}


\setlength{\headheight}{15.2pt}
\setlength{\headsep}{20pt}
\pagestyle{fancyplain}

\renewcommand{\ExerciseHeader}{\textbf{Problem \ExerciseHeaderNB .}}
\renewcommand{\AnswerHeader}{\textbf{Problem \ExerciseHeaderNB .}}

\newcommand{\be}[1]{\begin{Exercise}[label=#1]}
\newcommand{\ee}{\end{Exercise}}
\newcommand{\ba}[1]{\begin{Answer}[ref=#1]}
\newcommand{\ea}{\end{Answer}}
\newcommand{\ds}{\displaystyle}
\newcommand{\is}{\hspace{2pt}}
\newcommand{\dx}{\is dx}

\newcommand{\Z}{\mathbb{Z}}
\newcommand{\Q}{\mathbb{Q}}
\newcommand{\R}{\mathbb{R}}
\newcommand{\C}{\mathbb{C}}
\newcommand{\F}{\mathbb{F}}


\fancyhead[L]{Skills Check 2 Guide}
\fancyhead[C]{}
\fancyhead[R]{Math 240, Fall 2016}


\begin{document}

For Skills Check 2, you should be able to give a mathematically precise definition of the following:
\begin{itemize}	
	\item Linear transformation (in general)
	\item Inverse of a linear transformation
	\item Inverse of a matrix
	\item Elementary matrix
	\item Rank of a matrix
	\item Subspace
	\item Basis
	\item Dimension
	\item Range
	\item Null space
	\item Row space
	\item Column space
\end{itemize}

You should be able to state the following theorems:
\begin{itemize}
	\item The Rank-Nullity Theorem
	\item At least three conditions equivalent to an $n\times n$ matrix being invertible.
		(see The Invertible Matrix Theorem)
\end{itemize}

You should be able to do the following computations:
\begin{itemize}	
	\item Given a linear transformation $T$, compute $T(\vec{v})$.
	\item Given a linear transformation $T$, find a matrix for $T$.
	\item Find the inverse of a matrix.
	\item Use several different methods to determine whether or not a matrix is invertible.
	\item Determine whether or not a transformation is linear.
	\item Determine whether or not something is a subspace.
	\item Given a subspace, find a basis for it and compute its dimension.
	\item Describe the range and null space of a linear transformation.
	\item Give a basis for the range and null space of a linear transformation.
	\item Describe the row space, column space, and null space of a matrix.
	\item Give a basis for the row space, column space, and null space of a matrix.
\end{itemize}

All of these are in addition to the items on Skills Check 1.



\end{document}
